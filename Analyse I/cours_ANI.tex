\documentclass[a4paper, oneside]{report}
\usepackage[top=3cm, bottom = 3cm, left = 3cm, right = 3cm]{geometry}
\usepackage{amsfonts,amsmath,amssymb}
\usepackage[utf8]{inputenc}
\usepackage[francais]{babel}
\usepackage{graphicx}
\usepackage{polynom}
\usepackage[T1]{fontenc}
\usepackage{mathenv}
\usepackage{mdwtab}
\usepackage{array}
\usepackage{tikz} % \begin{tikzpicture}
\usepackage{pdfpages} %\includepdf[page={1-5}]{truc.pdf}
\usepackage[colorlinks=true,linkcolor=black]{hyperref}

\newcommand{\x}{\times}
\newcommand{\R}{\mathbb{R}}
\newcommand{\Rb}{\bar{\R}}
\newcommand{\N}{\mathbb{N}}
\newcommand{\K}{\mathbb{K}}
\newcommand{\C}{\mathbb{C}}
\newcommand{\D}{\mathbb{D}}
\newcommand{\Z}{\mathbb{Z}}
\newcommand{\Q}{\mathbb{Q}}
\newcommand{\displayastyle}{\displaystyle}
\newcommand{\sev}{sous-espace vectoriel }
\newcommand{\sevs}{sous-espaces vectoriels }
\newcommand{\ev}{espace vectoriel }
\newcommand{\et}{espace topologique }
\newcommand{\evn}{espace vectoriel normé }
\newcommand{\defi}{\subsubsection{Définition}}
\newcommand{\dem}{\subsubsection{Démonstration}}
\newcommand{\propr}{\subsubsection{Propriété}}
\newcommand{\propo}{\subparagraph{Proposition}}
\newcommand{\remar}{\subsubsection{Remarque}}
\newcommand{\exem}{\subsubsection{Exemple}}
\newcommand{\cor}{\subsubsection{Corollaire}}
\newcommand{\theo}{\subsubsection{Théorème}}
\newcommand{\fracun}[1]{\frac{1}{#1}}
\newcommand{\cerc}[1]{\overset{\circ}{#1}}

\begin{document}

\title{Cours d'analyse I}
\date{11/09/2018}
\author{Patrice Perrin}
\maketitle

\tableofcontents{}
\chapter{Espace vectoriel normé}

\section{Espace vectoriel normé et autres}

À un \ev normé, on va cherche à ajouter une structure topologique pour savoir si 2 points sont proches.

\defi
Soit E un \ev sur $\R$. Soit N:E$\rightarrow \R+$ est une norme sur E ssi :

\begin{enumerate}
\item $N(u)=0 \Leftrightarrow u=0$
\item $N(\lambda u)=|\lambda| N(u)\hspace{1em}, \lambda\in \R, u\in E \hspace{5em}$ (homogène)
\item $N(u+v)\leq N(u)+N(v)$ \hspace{5em} (inéquation triangulaire)
\end{enumerate}
On dit alors que $(E, N)$ est un \evn.

\defi
\label{def-evm}
Soient X un ensemble et $d:X^2\rightarrow \R+$ appelé distance.

\begin{enumerate}
\item $d(x,y) = 0 \Leftrightarrow x=y$
\item $d(x,y)=d(y,x)$\hspace{1em}$x,y\in X$
\item $d(x,z)\leq d(x,y)+d(y,z)$\hspace{1em}$x,y,z\in \R$
\end{enumerate}
On dit alors que $(X,d)$ est un espace métrique.


\subsubsection{Définition}
$n$ est dite semi-normé sur E (\ev de $\R$) ssi n vérifie 2 et 3 de la définition \ref{def-evm}.

\subsubsection{Abstract nonsense}
\begin{enumerate}
\item $N(-u)=N(u)$
\item $N(\lambda u)=0 \Leftrightarrow \lambda u =0 \Leftrightarrow \lambda =0~ou~u=0$
\item $N(\sum_{i=1}^{n}u_i) \leq \sum_{i=1}^{n}N(u_i)$
\item $|N(u)-N(v)|\leq N(u-v)$
\end{enumerate}

\subsubsection{Définition}

Soit $(E,N)$ ou $(X,d)$, alors pour $x_0\in E \text{ (ou }X)$ :
$$\left\{\begin{array}{lll}
(boule~ouverte)~B(x_0,r)&=& x\in E(ou~X), N(x-x_0)<r,~ou~d(x_0,x)<r\\
(boule~fermée)~\bar{B}(x_0,r)&=&x\in E(ou~X), N(x-x_0)\leq r,~ou~d(x_0,x)\leq r\\
\end{array}\right.$$

\subsubsection{Remarque}
Tout \evn est un espace métrique pour :
$$d_N(x,y)=N(y-x)\hspace{1em}x,y\in E$$

\subsubsection{Exercice}
Il y a des distances sur un \ev qui ne proviennent pas de normes (distance discrète) :
$$d(x,y)=\left\{\begin{array}{ll}
0 & x=y\\
1 & x\neq y
\end{array}\right.$$

\subsubsection{Définition}
Soit U une partie de $(E,N)$ ou de $(X, d)$ est dite ouverte ssi :
$$\forall x_0\in U, \exists r>0~tq~B(x_0,r)\subset U$$

\subsubsection{Définition}
Une partie $V$ de $(E,N)$ ou $(X, d)$ est un voisinage de $x_0\in E~(ou~X)$ ssi il existe U ouvert contenant $x_0$ et contenu dans V.

\subsubsection{Remarque}
L'ensemble des ouverts de E (ou X) comprend :
\begin{itemize}
\item $\emptyset$ de E( ou de X)
\item toute réunion d'ouverts est encore ouverte
\item toute intersection finie d'ouverts est encore ouverte
\end{itemize}

\subsubsection{Définition}
On appelle espace topologique $(X, \mathcal{T})$ un ensemble muni d'une famille de parties $T$ ($\subset P(x)$) dites ouvertes qui vérifie :
\begin{enumerate}
\item $\emptyset$, $X\in mathcal{T}$
\item $ \forall \alpha\in J, U_\alpha \in mathcal{T} \Rightarrow \bigcup_{\alpha \in J}U_\alpha \subset \mathcal{T}$ \\
  (ie. Toute union -- pas nécessairement dénombrable -- d'ouverts de $X$ est un ouvert de $X$)
\item $\forall i\in I$ (I de cardinal fini) $U_\alpha\in T \Rightarrow \bigcap_{i\in I}U_i\subset T$ \\
  (ie. Toute union \textbf{finie} d'ouverts de $X$ est un ouvert de $X$)
\end{enumerate}

\subsubsection{Propriété}
Par cette définition, tout \evn $(E, N)$ (tout espace métrique $(X,d)$) est muni d'une topologie (la topologie associé à la norme ou à la distance).\\
Une topologie ne provient pas nécessairement d'une métrique (à fortiori une norme).\\
Il y a des conditions nécessaires et suffisantes pour qu'une topologie provienne d'une métrique (on dit métrisable).

$$ espace~vectoriel~normé \Rightarrow espace~métrique \Rightarrow espace~topologie $$
Les réciproques sont fausses.

\subsubsection{Remarque}
Une boule ouverte est ouverte.

\subsubsection{Remarque}
Il existe des distances ultra-métriques où la distance est :
$$d(x,z)\leq max(d(x,y),d(y,z))$$

\subsubsection{Définition}
On appelle fermé dans un espace topologique le complémentaire d'une partie ouverte :
$$F~\text{fermé} \Leftrightarrow X\backslash F~est~ouvert$$

\subsubsection{Remarque}
$\bar{B}(x_0,r)$ est fermé.

\subsubsection{Définition}
A$\subset$ ($(X,T), (X,d), (E,N)$)\\
$\cerc{A}$ (intérieur de A), le plus grand ouvert dans A\\
$\bar{A}$ (adhérence de A), le plus petit fermé contenant A\\
$Fr(A)= \bar{A} \backslash \cerc{A}$ (frontière de A) et $\cerc{A}\subset A \subset \bar{A}$\\

\subsubsection{Exemple}
Soit $]0,1[ \subset [0,1[ \subset [0,1]$, on a $Fr([0,1[)=\{0,1\}$

\subsubsection{Définition}
Soit $f$ de $(E,N_E)$ dans $(F,N_F)$ est continue en $x_0$ ssi :
$$\forall\epsilon >0~\exists \eta>0~N_E(x-x_0)< \eta \Rightarrow N_F(f(x)-f(x_0))<\epsilon$$
f est continue de E dans F ssi f est continue en tout point de E.

\subsubsection{Définition}
A est dite bornée (dans (E,N)) ssi :
$$\exists M>0~\exists x_0 \in E~tq~A\subset B(x_0,M)$$

\subsubsection{Définition}
On dit que $N_1$ et $N_2$ deux normes sur un même \evn E sont équivalentes ssi :
$$\exists m>0~ \exists M>0~tq~\forall u\in E~mN_1(u)\leq N_2(u)\leq MN_1(u)~et~\fracun{M}N_2(u)\leq N_1(u)\leq \fracun{m}N_2(u)$$

\subsubsection{Remarque}
Si $x=0$, on a :
$$m0\leq 0 \leq M0$$
et si $x\neq 0$, on a :
$$m \leq \frac{N_2(x)}{N_1(x)} \leq M$$
et $M$ est donc le plus petit possible et $m$ le plus grand possible.


\subsubsection{Remarque}
(E,$N_1$) et (E,$N_2$) ont la même topologie.

\subsubsection{Proposition}

Soient $(E_1,N_1)$ et $(E_2,N_2)$ deux espaces normés, $E_1 \x E_2$ peut être muni de la norme :
\begin{itemize}
\item $(x,y) \mapsto \sup (N_1(x), N_2(y))=N(x,y)$
\item ou $(x,y) \mapsto N_1(x)+N_2(x)=N'(x,y)$
\end{itemize}
Ces deux normes sont équivalentes et définissent la même topologie sur $E_1\x E_2$.

\subsubsection{Démonstration}

On montre que la norme est nulle si les deux sont nuls, la norme du produit par un scalaire et le produit du scalaire avec la norme, et enfin on a l'inégalité triangulaire.\\
Les deux normes sont équivalentes car :
$$N(x,y) \leq N'(x,y) \leq 2N(x,y)$$

\subsubsection{Exemple}
$(\lambda , x)\mapsto \lambda x$ est continue de $\R \x E$ dans $E$.\\
$(x , y)\mapsto x+y$ est continue de $E \x E$ dans $E$.\\

\subsubsection{Remarque}
Soit $E$ un \ev et $T$ une topologie sur $E$, alors $(E,T)$ est un \ev topologique ssi :
\begin{itemize}
\item $(x,y) \mapsto x+y$
\item $(\lambda , x) \mapsto \lambda x$
\end{itemize}

\section{Topologie et norme / distance}

\subsubsection{Propriété}

$f$ est continue pour la topologie associée à une norme ou à une distance est équivalent à $f$ est continue pour la norme ou la distance.

\subsubsection{Propriété}
Soit $A \subset (E,N)$ (avec $E=X~ou~E$ et $N$ une norme ou une distance). On a :
\begin{enumerate}
\item $\cerc{A} = \left\{a \in A;~\exists r >0~B(a,r)\subset A \right\}$
\item $\bar{A} = \left\{x\in E;~\forall r>0~ B(x,r) \cap A \neq \emptyset \right\}$
\item $x\in Fr(A)~\Leftrightarrow~\left\{x\in E;~ \exists r>0~ B(x,r)\cap A\neq \emptyset~et~ B(x,r)\cap (X\backslash A)\neq \emptyset \right\}$
\end{enumerate}

\subsubsection{Démonstration}

\begin{enumerate}
\item $\supseteq$ : Si $a\in A$ et $\exists r>0~B(a,r)\subset A$. Or $B(a,r)$ est un ouvert donc par définition de l'ouverture $a\in \cerc{A}$\\
$\subseteq$ : Si $a\in \cerc{A}$, $\exists U$ ouvert, $U\subset \cerc{A}$, avec $a\in U$. D'après la caractérisation des ouverts d'un espace métrique $\exists~r>0$ $B(a,r)\subset U$

\item Si $a\in A$, la propriété est claire.\\
  Si $a\notin A$ (ie. $x \in X\backslash A$):
  dire que 
  $$X\backslash \bar{A} = \left\{x\in X;~\exists r>0~ B(x,r) \cap A = \emptyset \right\}$$
  est équivalent à dire
  $$X\backslash \bar{A} = \left\{x\in X;~\exists r>0~ B(x,r) \subset X\backslash A \right\}$$
  
  S'il existe un $r$ tel que $B(x,r)\subset X\backslash A$, alors $F=X\backslash B(x,r)$ est un fermé tel que $A\subset F$. Or l'adhérence de $A$ est le plus petit fermé contenant $A$, donc $\bar{A}\subset F$. Puisque $x\in B(x, r)$, $x\notin F$, c'est à dire $x\in X\backslash \bar{A}$\\
  Si $x\in X\backslash \bar{A}=X\ \bigcap_{F\supset A}F$ par définition de l'adhérence dans un espace topologique. $x$ est dans le complémentaire de l'intersection de fermé donc un fermé; on en déduit que $x$ est dans un ouvert n'intersectant pas $\bar{A}$. D'après la caractérisation des ouverts dans les espaces métriques, il existe un $r>0$ tel que  $B(x, r) \subset X\backslash \bar{A}$.\\
  
\item Si $x\in Fr(A)=\bar{A} \backslash \cerc{A}$, alors 
\begin{itemize}
\item  $\forall r>0~B(x,r)\cap A \neq \emptyset$ ($\in \bar{A}$)
\item $x\in \cerc{A}$ $\forall r>0~B(x,r)\nsubseteqq A$ et $\forall r>0~B(x,r)\cap (X\backslash A)\neq \emptyset$
\end{itemize}

\end{enumerate}

\subsubsection{Définition}
On dit que $x$ est un point d'accumulation pour la partie $A$ ssi $\forall r>0~(B(x,r)\backslash \{x\}) \cap A \neq \emptyset$ (boule centrée en x épointée).\\
Si $x\in A$ ne vérifie pas cette propriété, on dit que $x$ est un point isolé de $A$.

\subsubsection{Remarque}
Si $x$ est un point d'accumulation :
\begin{itemize}
\item $x$ peut appartenir à $A$
\item sinon $x\notin A$ et $x\in \bar{A}$
\end{itemize}

\subsubsection{Exemple}

$\left\{x\in \R;~x=\fracun{n},~ n \geq 1 \right\}$, $\{0\}$ est un point d'accumulation de $A$ et tous les points de $A$ sont isolés.

\subsubsection{Propriété 1}

Si $(E,N)$ est un espace normé alors $\forall r>0$ $\overline{B(x,r)}=\bar{B}(x,y)$ (l'adhérence de la boule ouverte est la boule fermée)

\subsubsection{Démonstration}

Soit $x\in E$, on a $B(x,r) = \{y, N(y-x) <r \} \subset E$.\\
On sait que $B(x,r)\subset \overline{B(x,r)}\subset \bar{B}(x,y)$.\\
Soit $y$ tq $N(y-x)=r$.
On a :
$$z\in [x,y] \Leftrightarrow z=(1-\lambda)x+\lambda y\hspace{1em}\lambda \in [0,1]\hspace{5em} \text{(\ev)}$$
Il suffit de vérifier qu'il existe $z$ avec 
\begin{itemize}
\item $z\in B(x,r)$
\item $z\in B(y,\rho)$

\end{itemize}

$$\begin{array}{lll}
N(z-x)&=& N(\lambda y + (1-\lambda)x -x)\\
&=&|\lambda| N(y-x)= \lambda r
\end{array}$$
et
$$\begin{array}{lll}
N(z-y)&=& N(\lambda y + (1-\lambda)x -y)\\
&=&|1- \lambda| N(y-x)= (1-\lambda) r
\end{array}$$

Si $\lambda \in [0,1[$ alors $z\in B(x,r)$ et $z\in B(y,\rho)$, donc $(1-\lambda)r <\rho$.\\
Donc si on prend $\lambda$ tq $(1-\lambda) < \frac{\rho}{r}$ ($\lambda \neq 1$ mais assez proche de 1), la propriété est vérifié.

\subsubsection{Propriété 2}

Si $(E,N)$ est un \ev normé et $F$ un sous-espace vectoriel, alors $\bar{F}$ est un \sev.

\subsubsection{Démonstration}

Soit $(E,N)$ et $F$ un \sev de E. 

Il est clair que $\bar{F}$ est une partie fermée. Il suffit donc de montrer que $\bar{F}$ est stable par combinaison linéaire. Nous allons utiliser la caractérisation de l'adhérence donnée précédemment. \\

Montrons, dans un premier temps, que $\bar{F}$ est stable par multiplication par un scalaire. 

Soit $\lambda \in \R$ et $x \in \bar{F}$. Si $\lambda = 0$, le point $\lambda x$ appartient à $F$ donc à $\bar{F}$. Supposons donc $\lambda \neq 0$.

Soit $B(\lambda x, r)$ une boule centrée en $\lambda x$ de rayon $r > 0$. Puisque $x$ est adhérent à F, il existe $z$ dans la boule $B(x,\frac{r}{|\lambda|})$. On a alors :
$$N(\lambda z - \lambda x) = |\lambda|N(z-x) < |\lambda|\frac{r}{|\lambda|} = r$$
Donc le point $\lambda z$ appartient à $B(\lambda x, r)$ et, $r$ étant quelconque, le point $\lambda x$ est bien adhérent à F. \\

Montrons maintenant la stabilité pour l'addition.

Soit $r>0$, $x,y\in \bar{F}$ et $B(x+y,r)$ une boule centrée en $x + y$. Comme $x$ est adhérent à $F$ , on sait que
la boule $B(x, \frac{r}{2})$ contient un point $z$ de $F$. Il en est de même pour $B(y, \frac{r}{2})$ qui contient un point $t$ de $F$. On a donc, par inégalité triangulaire : $$N(z+t-x-y)< N(z-x)+N(t-y)<\frac{r}{2} + \frac{r}{2} = r$$
Le point $z + t$ est donc dans la boule $B(x + y, r)$ et le point $x + y$ est bien adhérent à $F$.\\

Finalement, $\bar{F}$ est bien un \sev.

\subsubsection{Remarque}
\begin{enumerate}
\item La propriété 1 est fausse en générale pour les espaces métriques
\item On se restreindra souvent aux \sev fermés.
\end{enumerate}

\subsubsection{Définition}

La suite $x_n$ (dans un \ev normé ou un espace métrique) est convergente vers l ssi :
$$\forall \epsilon >0~ \exists n_0>0~tq~n\geq n_0 \Rightarrow N(l-x_n)<\epsilon~(ou~d(l,x_n)<\epsilon)$$

\subsubsection{Abstract nonsense}

\begin{enumerate}
\item Si l existe, elle est unique
\item Si $u_n \rightarrow l$ et $v_n\rightarrow l'$, $u_n+v_n \rightarrow l+l'$
\item Si $u_n \rightarrow l$ et $\lambda \in \R$, $\lambda u_n \rightarrow \lambda l$
\end{enumerate}

\subsubsection{Propriété}
Si $u_n\rightarrow l$ alors $u_n$ est bornée

\subsubsection{Démonstration}
Soit $(u_n)$ une suite convergeant vers $l$.
Alors à partir d'un certain rang $n_0$ tout les termes de la suite sont dans la boule centrée en l de rayon 1.
Puisqu'il y a un nombre fini de termes avant $n_0$, on sait que l'ensemble $\{x_n, n < n_0\}$ est borné.
Puisque l'ensemble $\{x_n,  n < n_0\}$ est borné (car il y a un nombre fini de terme) et que l'ensemble des $\{x_n,  n > n_0\}$ est aussi borné, l'ensemble de tous les $x_n$ est borné.

\subsubsection{Définition}
La suite $(u_n)$ est de Cauchy ssi :
$$\forall \epsilon >0~ \exists n_0>0~tq~n,m\geq n_0 \Rightarrow N(u_n-u_m)<\epsilon~(ou~d(u_n,u_m)<\epsilon)$$
ou 
$$\forall \epsilon >0~ \exists n_0>0~tq~n\geq n_0 \Rightarrow \forall p~N(u_{n+p}-u_n)<\epsilon~(ou~d(u_{n+p},u_n)<\epsilon)$$
Si $u_n$ est convergente, elle est de Cauchy.

\subsubsection{Propriété}

Si $u_n \rightarrow l$ ou si $u_n$ est de Cauchy, alors $\{u_n,~n\geq 0 \}$ est borné.

\subsubsection{Démonstration}
On sait qu'à partir d'un certain rang $n_0$, $N(u_n) \leq l+1$ $n\geq n_0$, et $\{u_i, i\leq n_0 \}$ est un ensemble fini, donc $\{u_n,~n\geq 0 \}$ est borné par le sup entre $\{N(u_i), i\leq n_0 \}$ et $N(l)+1$.

\subsubsection{Définition}

$(X,d)$ est complet ou $(E,N)$ est de Banach (espace vectoriel normé complet), ssi toute suite de Cauchy est convergente.

\subsubsection{Exemple / contre-exemple}

\begin{enumerate}
\item $(\Q, |.|)$ n'est pas complet (n'admet pas la propriété de la borne supérieur).
\item $(\R, |.|)$ est un corps ordonné contenant $\Q$ admettant la propriété de la borne supérieure.\\
Soit $u_0,...,u_n$ une suite de Cauchy bornée. On a :
$$m\leq Inf(u_0,...,u_n)\leq Sup(u_0,...,u_n)\leq M$$
$$m\leq Inf(u_1,...,u_n)\leq Sup(u_1,...,u_n)\leq M$$
On pose $\sigma_n=Inf_{i\geq n}\{u_n\}$ et $\tau_n=Sup_{i\geq n}\{u_n\}$.\\
On a $\sigma_n$ croissante et $\tau_n$ décroissante, de plus $\sigma_n \leq \tau_n$, et aussi $\tau_n - \sigma_n \rightarrow 0$ (suite monotone adjacente).\\
Donc $\sigma_n \rightarrow Sup\sigma_n$ et $\tau_n \rightarrow Inf \tau_n$ et $Sup~\sigma_n = Inf~\tau_n$.

\paragraph{Commentaire}
Construire $\R$ à partir de $\Q$ :
\begin{itemize}
\item "Coupure de Dedekind" (partition de $\Q$ avec $\Q=E\sqcup F$ avec $e\in E \leq f\in F$).\\
On prend $E=\Q \cup \{r\geq 0, r^2<2 \}$ et $F=\{r\geq 0, r^2>2 \}$

\item $(\Q, |.|)$ est un \evn. On le complète par {suites de Cauchy de rationnels} et {suites de Cauchy tendant vers 0}.
\end{itemize}

\item $(\R_n, ||.||_\infty)$ (on sait que $||.||_\infty$ est équivalente à $||.||_p$ $\forall p$, donc elles ont les mêmes suites de Cauchy) est complet (Banach).\\
Soit $x_m \in \R^n$ avec $x_m=(x_1^m,...,x_n^m)$. Elle est de Cauchy car $|x_i^{m+p}-x_i^m| \leq ||x_{m+p}-x_m||_\infty$.\\
Donc chaque $x_i^m$ est de Cauchy dans $\R$.

\item $\R[X]$ avec 
$$||P||_2 = \left\{\begin{array}{lll}
0 &si& P=0\\
\sqrt{\sum_{0}^{\deg P}a_i^2} &si& P\neq 0
\end{array}\right.$$
avec $P=a_0+...+a_nX^n$.\\
On a $P_n=1+\sum_{i=1}^n\frac{x^i}{i}$.\\
$$||P_{n+p}-P_n|| = \sqrt{\sum_{i=n+1}^{n+p}\frac{1}{i^2}}$$
$$||P_{n+p}-P_n|| \leq \sqrt{\sum_{i=n+1}^{+\infty}\frac{1}{i^2}}$$
Le membre de droite est le reste d'une série de Riemann convergente, il converge donc vers 0.\\
C'est une suite de Cauchy.\\
Quelle pourrait être la limite ?\\
Ça ne peut pas être 0 car $||P_n-0||_2=||P_n||_2=\sqrt{1+\sum_{i=1}^{n}\frac{1}{i^2}}$ qui ne tend pas vers 0.\\
Si $Q$ est un polynôme : $||P_m-Q|| = \sqrt{(1-q_0)^2+...+(1-q_n)^2+\sqrt{\sum_{i\geq n}^{m}\frac{1}{i^2}} }$. Si $m>n$ alors cela ne converge pas car il restera le degré le plus élevé.\\
Donc cet espace n'est pas complet pour $||.||_2$

\item $C([0,1],\R)$ (ensemble des fonctions continues de [0,1] dans $\R$) est complet pour $||.||_\infty$. On a $||f||_\infty = Sup_{x\in [0,1]}|f(x)|$ (la norme est bien définie sur l'espace).\\
\begin{enumerate}
\item Trouver un candidat pour la limite
\item Montrer que la suite tend vers le candidat
\item Montrer que le candidat est dans l'espace
\end{enumerate}

Soit $f_n\in C([0,1],\R)$ de Cauchy.
$$Sup_{x\in [0,1]} |(f_{n+p}-f_n)(x)| = ||f_{n+p}-f_n||_\infty \rightarrow 0$$
\begin{enumerate}
\item Si $x\in [0,1]$, $f_n(x)$ est une suite de Cauchy de réel, elle tend vers $\varphi(x)$ (convergence uniforme, donc convergence simple ou ponctuelle).\\
\item On doit monter que pour $\epsilon >0$, il existe $n$ assez grand tq $\forall x\in [0,1]$ $|\varphi(x)-f_n(x)|<\epsilon$.\\
$$|\varphi(x)-f_n(x)| \leq |\varphi(x)-f_m(x)|+|f_m(x)-f_n(x)|$$
$$|\varphi(x)-f_n(x)| \leq  |\varphi(x)-f_m(x)|+||f_m(x)-f_n(x)||_\infty$$
$\forall \epsilon >0,~ \exists n_0>0~tq~||f_m(x)-f_n(x)||_\infty < \frac{\epsilon}{2}$.\\
Pour cet $x$, $\exists m(x)\geq n_0$ et $>>0$ tq $|\varphi(x)-f_m(x)|< \frac{\epsilon}{2}$.\\
D'où $|\varphi(x)-f_n(x)|< \epsilon$ si $n>n_0$.\\
\item Il faut montrer que $\varphi$ est continue.
$$|\varphi(x)-\varphi(x_0)|\leq  |\varphi(x)-f_n(x)|+|f_n(x)-f_n(x_0)|+|f_n(x_0)-\varphi(x_0)|$$
$$|\varphi(x)-\varphi(x_0)|\leq 2 ||\varphi - f_n||_\infty + |f_n(x)-f_n(x_0)|$$
$\exists n_0$ tq $n>n_0$ $||\varphi -f_n||_\infty <\frac{\epsilon}{3}$.\\
De plus, pour un tel $n$, $f_n$ est continue en $x_0$ d'où $\exists n>0$ tq $|f_n(x)-f_n(x_0)|<\frac{\epsilon}{3}$
Donc $|\varphi(x)-\varphi(x_0)|\leq \epsilon$.
\end{enumerate}
\end{enumerate}

\subsubsection{Propriété}

Soit $(E,N)$ de Banach, alors une partie est complète ssi elle est fermée.\\

\subsubsection{Corollaire}
Si F est un \sev de $(E,N)$ de Banach, alors $\bar{F}$ est de Banach (on sait que $\bar{F}$ est un \sev normé).

\subsubsection{Démonstration}
$F$ est fermée dans $E$. Soit $x_n$ une suite de Cauchy dans $F$, donc $x_n$ est de Cauchy dans $E$.\\
$x_n$ tend vers $x \in E$ et $x\in \bar{F}$. Donc $B(x,r)$ est fermée : $\exists n>>0$ tq $x_n\in B(x,r)$.\\
Si $F$ est complet, $x\in \bar{F}$ alors $B(x,\fracun{n}) \cap F \neq 0$, d'où $x_n \in F\cap B(x,\fracun{n})$.\\
D'où $x_n\in F$, $x_n\rightarrow x$. $x_n$ est de Cauchy et donc $x\in F$.


\subsubsection{Remarque}
\begin{enumerate}
\item $\bar{F}$ se caractérise aussi comme les points limite d'une suite $x_n\in F$.\\

Si $f$ est continue en $x_0$ :
$$\forall \epsilon >0~\exists n>0~N_E(x-x_0)<n\Rightarrow N_F(f(x)-f(x_0)) < \epsilon$$
Si $x_n\rightarrow x_0$ $\exists n_0$, $N_E(x-x_0)<n_0$ d'où $N_F(f(x)-y_0) < \epsilon$

\item $f$ est de $E$ dans $F$ continue en $x_0$ ssi pour toute suite $x_n\rightarrow x_0$ $f(x_n)\rightarrow f(x_0)=y_0$\\

Si $\forall$ suite $x_n\rightarrow x_0$ $f(x_n)\rightarrow y_0$. Par l'absurde, on a :
$$\exists \epsilon >0~\forall n~\exists x~ N_E(x-x_0)<n~et~N_F(f(x)-y_0)\geq \epsilon$$
On prend $n=\fracun{n}$ et $x=x_n$, on a : $N_E(x_n-x_0)<\fracun{n}$ et $N_F(f(x_n)-y0)>\epsilon$. Donc il existe une suite $x_n$ tq $f(x_n)$ ne tend pas vers $y_0$.

\end{enumerate}

\section{Application des \ev de Banach}

\subsection{Théorème du point fixe}

\subsubsection{Définition}

$f$ est une fonction de $(E,N_E)$ dans $(F,N_F)$ est k-contractante ssi $\exists k\in [0,1[$ tq :
$$x,y\in E~N_F(f(x)-f(y))\leq kN_E(x-y)$$
Si $f$ est contractante, alors elle est continue.
\subsubsection{Théorème}

Soit $f:(E,N) \rightarrow (E,N)$ k-contractante ($0< k<1$) et $(E,N)$ de Banach, alors $f$ admet un unique point fixe (tq $f(x)=x$).

\subsubsection{Remarque}

Si $N(f(x)-f(y))=0$ c'est que $f$ est une fonction constance.

\subsubsection{Démonstration}

Soit $x_0\in E$ et $x_{n+1}=f(x_n)$ ou $x_n=f^n(x_0)$.\\
$$N(x_{n+1}-x_n)=N(f(x_n)-f(x_{n-1}))< kN(x_n-x_{n-1})$$
Par récurrence, on obtient :
$$N(x_{n+1}-x_n) \leq k^nN(f(x_0)-x_0)$$
Donc :
$$N(x_{n+p}-x_n)\leq N(x_{n+p}-x_{n+p-1})+...+N(x_{n+1}-x_n)$$
$$N(x_{n+p}-x_n)\leq (k^{n+p}+...+k^n)N(f(x_0)-x_0)$$
$$N(x_{n+p}-x_n)\leq \frac{k^n(1-k^{p+1})}{1-k} N(f(x_0)-x_0)$$
$$N(x_{n+p}-x_n)\leq \frac{k^n}{1-k}N(f(x_0)-x_0)$$
Le membre de droite tend vers 0 si $n$ tend vers $+\infty$.\\
Donc $x_n$ est de Cauchy. Donc $x_n\rightarrow l \in E$.\\
On a $f(x_n)=x_{n+1}\rightarrow f(l)=l$ (si $f$ est continue et c'est vrai car $f$ est contractante).\\
Mais un point fixe est unique :\\
Soit $l_1,l_2$ deux points fixes de $f$, on a :
$$N(f(l_1)-f(l_2)) \leq kN(l_1-l_2)$$
$$N(l_1-l_2) < N(l_1-l_2)$$
Ce qui est impossible. Donc le point fixe est unique.

\subsection{Séries dans la Banach}

Soit $(E,N)$, un \evn.\\

\subsubsection{Définition}
L'étude d'une série $(u_i)$ est l'étude de $U_i=\sum_{j=0}^{i}u_j$, suite des sommes partielles.\\
La série est convergente ssi la suite $U_i$ est convergente.\\
La série est de Cauchy ssi la suite $U_i$ est de Cauchy.\\
Une série est normalement convergente ssi les séries des $N(u_i)$ est convergente.

\subsubsection{Propriété}
Une série normalement convergente est convergente (si $E$ est de Banach).\\
Si $\sum_{0}^\infty N(u_i)$ est définie alors $\sum_{0}^\infty u_i$ est définie dans $E$.

\subsubsection{Démonstration}
On a :
$$\begin{array}{lll}
N(U_{n+p}-U_n)&=&N(\sum_{i=1}^p u_{n+i})\\
&\leq & \sum_{i=1}^p N(u_{n+i})\\
&\leq & \sum_{i=1}^\infty N(u_{n+i})\\
&\rightarrow & 0
\end{array}$$

Donc $U_i$ est de Cauchy et donc est convergente ($E$ est de Banach).

\subsubsection{Remarque}
$N(\sum_{0}^\infty u_i) \leq \sum_{0}^\infty N(u_i)$

\subsubsection{Exercice}
$C([0,1], \R)$ est de Banach par $||.||_\infty$.\\
Si $||\varphi_i||_\infty$ est une série convergente, alors $x\mapsto \sum_{0}^\infty \varphi_i(x)$ a un sens : fonction continue en $x\in [0,1]$.

\section{Compacité}

Soient $(X,d)$ un espace métrique, $(E,N)$ un espace normé, et $(X,C)$ un espace topologique séparé.\\

\subsubsection{Définition}
Un \et est séparé ssi $\forall (x,y)\in X,~(x\neq y),$ $\exists U,V$ ouverts disjoints $x\in U, y\in V$ tq $(U\cap V = \emptyset)$.\\
Donc $(E,N)$ et $(X,d)$ sont séparés.


\paragraph{Recouvrement de K} est $U_\alpha$ une famille de parties tq $\cup_\alpha U_\alpha \supseteq K$

\subsubsection{Définition (Borel-Lebesgue)}
Soit $K\subseteq (X,C)$ séparé est dite compacte ssi pour tout recouvrement ouvert de $K$, on peut extraire un recouvrement fini de $K$ :
$$K\subseteq \cup_{\alpha \in A}U_\alpha \Rightarrow K\subseteq \cup_{i\in J}U_{\alpha_i}~(I~finie)$$

\subsubsection{Exemple}
\begin{enumerate}
\item $\R = \cup_n ]-n,n[$ est non compact (on peut le rendre compacte $\rightarrow$ $\bar{\R}=\R\sqcup \{\infty\}$ avec $x\in\R\mapsto e^{2i\pi x}\in \Pi=\R/\Z$ compactifié d'Alexandrov ou on peut le compacter avec $\R \sqcup \{+\infty\} \cup \{-\infty\}$, c'est la droite achevé).\\
\item $[0,1]$ (ou plus généralement $[a,b]$ fermé borné) est compacte.\\
On a $[0,1]\subseteq \cup_\alpha U_\alpha$ (par $|.|$).\\
Soit $\{t \in [0,1]; [0,t]\subseteq \sup_{j\in J}U_{\alpha_j} \}$ ($J$ est fini), est une partie fermée bornée et non vide car $0\in U_{\alpha_j}$.\\
Prenons $\tau=Sup\{t\in [0,1]; [0,t]\subseteq  \sup_{j\in J}U_{\alpha_j} \}$ ($J$ fini). On a $\tau \leq 1$ car $\tau \in U_{\alpha_\tau}$ ou $\tau$ est la borne supérieure d'où un point $t$ de l'ensemble des $]\tau - \eta_\tau, \tau[$.\\
Donc $[0,\tau]\subseteq [0,t]\cup ]\tau - \eta, \tau] \subseteq (\cup U_{\alpha_j}) \cup (U_{\alpha_\tau})$ et est un recouvrement de $[0,1]$.
\end{enumerate}

\subsubsection{Remarque}
Soit $K = \cup_\alpha (U_\alpha \cap K) \subseteq \cup_\alpha U_\alpha$ (un recouvrement), son complémentaire est $\emptyset = \cap_\alpha F_\alpha$ (fermés), alors une intersection finie (au moins) est déjà vide.

\subsubsection{Variante}
si une famille finie de fermés emboîtés est vide, alors au moins un de ses membres est vide.

\subsubsection{Contraposée de la variante}
Si $\forall n$ $F_n\neq \emptyset$ alors $\cap_n F_n \neq \emptyset$

\subsubsection{Proposition 1}
Si $K$ est compacte dans $(X,d)$ $(E,N)$ $(X,C)$, alors $K$ est fermée.

\subsubsection{Note}
Tout point $y \notin K$ peut être séparé de $K$ (c'est à dire qu'il existe deux ouverts disjoints, l'un contenant $\{y\}$ et l'autre contenant $K$).

\subsubsection{Proposition 2}
Si $K$ est compacte $(X,d)$ $(E,N)$, alors $K$ est bornée.

\subsubsection{Remarque}
Dans $(X,d)$ ou $(E,N)$ un ensemble compacte est fermé borné.\\

\subsubsection{Démonstration de la proposition 1}
Soit $k\in K$, et $y\in X\backslash K$.\\
$X$ est séparé car $B(k,r_k)\cap B(y,\eta_k)=\emptyset$.\\
$$K\subseteq \cup_k B(k,r_k)$$
$$K\subseteq \cup_{i\in I} B(k_i,r_{k_i}) ~~I\text{ finie}$$
$$\cap_{i\in I} B(y,\eta_i)=B(y,inf(\eta_i)\neq 0)$$
Et $B(y,inf(\eta_i))\cap (\cap_{i\in I} B(k_i,r_{k_i})) = \emptyset$ sinon $z\in B(k_i,r_{k_i})$ et $z\in B(y,\eta_i)$ $\forall i$, absurde.\\
On a donc $B(y,inf(\eta_i))$ $\subseteq X\backslash K$.\\
La démonstration est faite dans les espaces métriques, mais on a la même démonstration avec des ouverts.

\subsubsection{Démonstration de la proposition 2}
Soit $K\subseteq \cup_{i\in I} B(k_i,r_i)$, $k\in K$, donc :
$$N(k) \leq N(k-k_i)+N(k_i) \leq r_i+N(k_i)~(pour~un~i)$$
$$N(k) \leq Sup N(k_i) + Sup~r_i = M+R=M'$$

\subsubsection{Proposition}
Si $F \subseteq K$ est fermée dans $K$ alors $F$ est compacte (démonstration en exercice).

\subsubsection{Propriété (Bolzano-Weierstrass)}
Si $K$ est compacte (dans $(X,C)$), alors toute partie infinie $A$ de $K$ admet un point d'accumulation.

\subsubsection{Corollaire}
Si $K$ est compacte alors de toute suite de points de $K$, on peut extraire une suite convergente.

\subsubsection{Démonstration du corollaire}
Si $k_i$ est une suite de points de $K$. Soit $A=\{k\in K;\exists i~k=k_i \}$.\\
Soit $A$ est finie, et une des valeurs est atteinte une infinité de fois, d'où une suite stationnaire.\\
Soit $A$ est infinie, d'où (d'après le théorème) un point d'accumulation $\xi \in K$ tq $k_{\varphi(n)}\in B(\xi, \fracun{n})\backslash \{\xi\}$ et $k_{\varphi(n)} \rightarrow \xi$.\\

\subsubsection{Démonstration de la propriété}
Soit $A\subseteq K$ , $A$ infinie.\\
On montre que si $A$ n'a pas de point d'accumulation, alors $A$ est finie.\\
Si $A$ n'a pas de point d'accumulation :\\
$A$ est fermée (si $x\notin A$, il y a un voisinage ou une boule qui ne rencontre pas de point de $A$).\\
Donc $A$ est compacte.
$$A\subseteq \cup B(a,r)~avec~B(a,r)\backslash \{0\} \cap A = \emptyset$$
$$A\subseteq \cup_{i\in I} B(a_i,r_i)$$
Donc $A=\cup_{i\in I}\{a_i\}$ est finie.

\subsubsection{Propriété}
Si $K$ est compact dans $(X,d)$, $(E,N)$ alors $K$ est complet.

\subsubsection{Démonstration}
Soit $(k_n)$ une suite de Cauchy.\\
$\exists k\in K$ tq $k_{\varphi(n)} \rightarrow k$ (point d'accumulation de l'ensemble des valeurs de la suite).\\
Alors (exercice) une suite de Cauchy dont une sous-suite est convergente est elle-même convergente.\\
$K$ étant fermé, $k\in K$.

\subsubsection{Théorème}
Toute partie $K$ de $(X,d)$ ou $(E,N)$ est compacte (Borel-Lebesgue) ssi elle vérifie que toute partie infinie admet un point d'accumulation (Bolzano-Weierstrass).

\subsubsection{Lemme}
Si $K$ admet Bolzano-Weierstrass, alors pour tout recouvrement de $K$ par une famille d'ouverts $U_\alpha$, il existe $\xi > 0$ tq $k\in U_\alpha \Rightarrow B(k,\xi) \subseteq U_\alpha$.

\subsubsection{Remarque}
Un tel $\xi$ s'appelle membre de Lebesgue du recouvrement.

\subsubsection{Démonstration du Lemme}
Par l'absurde, on suppose que $K$ n'admet pas Bolzano-Weierstrass.\\
Pour $\xi = \fracun{2^n}$ $\exists k_n$ tq $\exists \alpha$ $k_n \in U_\alpha $ $B(k_n, \fracun{2^n}) \nsubseteq U_\alpha$.\\
$\{k_n\}$ est dans $K$ par hypothèse.\\
$\{k_{\varphi(n)}\}$ converge vers $k\in K$\\
$k\in K$ donc $k\in U_{\alpha_k}$, $B(k,\eta) \subseteq U_{\alpha_k}$\\
Donc $n\geq n_0$, $k_{\varphi(n)} \in B(k,\eta)$ voire $B(k,\fracun{2^{n+1}})$ $(\fracun{2^{n+1}} < \fracun{2^n}< \xi)$\\
Alors $k\in B(k_{\varphi(n+1)}, \fracun{2^{\varphi (n+1)}})$. Soit $x$ dans cette boule :
$$d(x,k) \leq d(x,k_{\varphi (n+1)}) + d(k_{\varphi (n+1)}, k) \leq \fracun{2^{n+1}} + \fracun{2^{n+1}}$$
Donc $x$ est dans $B(k,\fracun{2^n})$, donc il existe un point d'accumulation, on a enfin une contradiction.

\subsubsection{Démonstration du théorème}

Montrons que si on a BW, alors on a BL, c'est à dire $K$ vérifiant BW, alors $K\subseteq \cup_{\alpha \in A} U_\alpha$.\\
Par l'absurde, on suppose $K \nsubseteq \cup_{i\in I} U_{\alpha_i}$ pour toute partie finie $I$ de $A$.\\
Soit $\epsilon >0$, un nombre de Lebesgue associé à ($U_\alpha$).\\
On a $k_1\in K$ $B(k_1,\epsilon) \subseteq U_{\alpha_1}$, et $K\nsubseteq U_{\alpha_1}$.\\
De plus, $k_2\in K$ et $k_2\notin U_{\alpha_1}$ $B(k_2,\epsilon) \subseteq U_{\alpha_2}$, ce qui implique $d(k_1,k_2)\geq \epsilon$.\\
Par récurrence, on a $k_i\in B(k_i,\epsilon)\subseteq U_{\alpha_i}$ et $k_i\notin U_{j<i}U_{\alpha_j}$.\\
D'où, $k_{n+1}\in K$ mais $k_{n+1}\notin \cup_{i\leq n}U_{\alpha_i}$ et $d(k_{n+1}, k_i)\geq \epsilon$, $i\leq n$.\\
D'où $A=\cup \{k_n\}$ partie infinie sans point d'accumulation (tous ses points sont deux à deux à distance $\geq \epsilon$).

\propr
Tout fermé borné dans $\R$ (selon une topologie usuelle) ou $\R^n$ (idem) est compacte.

\dem
Pour $n=1$ (on le montre dans $\R$). Par BW, on a $A$ infinie $\subseteq F$ (fermé borné dans $\R$), $A\in [n,M]$.\\
On a $A_i\subseteq A_{i-1}\cdots \subseteq A$, et $A_n \subseteq $intervalle de Lebesgue $\frac{M-n}{2^n}$.\\
D'où une suite de points de $\R$ qui est de Cauchy.\\
Soit $a_n\in A_n\subseteq A \subseteq F$ tend vers $\alpha \in \R$. Mais $F$ est fermé d'où $\alpha \in F$, donc $A$ admet $\alpha$ comme point d'accumulation.\\
 
Pour $n\geq 2$. Soit $x_m = (x_1^m,...,x_n^m)$ $m\in \N$, une suite de points de $F$ fermé borné dans $\R^n$, d'où les $x_i^m$ sont dans un intervalle borné de $\R$.\\
On peut extraire des suites convergentes d'où : $\epsilon_i$ tq $x_i^{\varphi(m)}\underset{m \rightarrow +\infty}{\rightarrow }\epsilon_i$.\\
D'où une suite extraite $x^{\varphi(m)}\rightarrow \epsilon \in \R^n$ mais $F$ est fermée, donc $\epsilon \in F$.

\cor
$\bar{B}(x,r)$ est un compacte dans $\R^n$

\defi
$(X,C)$ séparé est dit localement compact ssi tout point $x\in X$ a un voisinage compact.\\
Ici $\R$ et $\R^n$ sont des espaces topologiques localement compacts.

\section{Applications continues}

\subsection{Propriétés locales}

\defi
$f:(E,N_E)\rightarrow (F,N_F)$ est continue en $x_0$ ssi :
$$\forall \epsilon >0~\exists n>0~N_E(x-x_0)<n\Rightarrow N_F(f(x)-f(x_0)) < \epsilon$$
ou ssi pour toute suite $x_n\rightarrow x_0$ $f(x_n)\rightarrow f(x_0)=y_0$

\propr
Si $f$ et $g$ sont continues en $x_0$, alors $\lambda f$ et $f+g$ sont continues en $x_0$ ($f,g$ de $(X,C)$ dans $(E,N)$).

\propr 
$f$ de ($E,N_E$) dans $(F,N_F)$ continues en $x_0$, et $g$ de $(F,N_F)$ dans $(G,N_G)$ continues en $f(x_0)$, alors $g\circ f$ est continue en $x_0$.

\subsection{Propriétés globales}

\defi
$f:(E,N_E)\rightarrow (F,N_F)$ est continue ssi $f$ est continue en tout point de $E$.

\propo
$f$ est continue de $(E,N_E)$ dans $(F,N_F)$ ssi $\forall V$ ouvert dans $F$ $f^{-1}(V)=U$ est un ouvert de $E$.\\
Ici, $f^{-1}$ est l'image réciproque de $f$.

\dem
On a $f^{-1}(V)=\{x\in E; f(x)\in V \}$.\\

Si $f$ est continue en tout point $x_0$ de $E$. Soit $V$ un ouvert de $F$. Si $y_0 \in V$, alors $\exists \epsilon >0$ $B(y_0,\epsilon)\subseteq V$.\\
Si $y_0 \notin f(E)$, il n'y a rien à démontrer.\\
Si $y_0\in f(E)$ :\\
Soit $x_0$ tq $f(x_0)=y_0$, alors $x_0\in f^{-1}(V)$ et $f$ continue en $x_0$ d'où $B(x_0,\eta) \subseteq f^{-1}(V)$ car $f(B(x_0,\eta))\subseteq B(y_0,\epsilon)$. Donc $f^{-1}(V)=U$ est un ouvert (voisinage de tous les points).\\

Soit $x_0\in E$, $y_0=f(x_0)$ $\forall \epsilon >0~B(y_0,\epsilon)$ ouvert de $F$ donc $f^{-1}(B(y_0,\epsilon))$ est un ouvert de $E$ centré en $x_0$.\\
D'où $B(x_0,\eta)$ tq $x_0 \in B(x_0,\eta)\subseteq f^{-1}(B(y_0,\epsilon))$

\exem
Par contre en général, l'image d'un ouvert n'est pas nécessairement un ouvert :
$x\mapsto x^2$ envoie $]-1,1[\subseteq\R$ sur $[0,1[\subseteq \R$

\remar
$f$ continue de $E$ dans $F$ ssi pour tout fermé $A$ de $F$ $f^{-1}(A)$ est fermé dans $E$.

\exem
Par contre l'image d'un fermé n'est pas nécessairement un fermé : $x\mapsto e^x$ envoie $]-\infty,0]$ (fermé de $\R$) sur $]0,1]$.

\propr
Si $f$ est continue de $(E, N_E)$ dans $(F, N_F)$ et si $K$ compact dans $E$ alors $f(K)$ est une partie compacte de $F$.

\dem
On a $f(K)\subseteq \cup_\alpha V_\alpha$.
$$K\subseteq f^{-1}(f(K)) \subseteq f^{-1}(U_\alpha V_\alpha)\subseteq U_\alpha f^{-1}(V_\alpha)$$
d'où $K\subseteq \cup_{i\in I} f^{-1}(V_{\alpha_i})$\\
d'où $f(K)\subseteq \cup_{i\in I} V_{\alpha_i}$

\exem
Par contre, $f^{-1}$ d'un compact n'est pas en général compact : $x\mapsto sin(x)$ $sin^{-1}([-1,1])=\R$

\propr
Tout fermé borné dans un espace de dimension finie (normé) est compact.

\subsubsection{Démonstration}
(pour $\R^n$ et $||.||_ \infty$ on sait)\\
Soient $E=<e1,...,e_n>=\R e_i \oplus ... \oplus \R e_n$, et $u\in E$.\\
$u = \sum_{i=1}^nx_ie_i \overset{b}{\leftarrow} x=(x_1,...,x_n)$ ($b$ une fonction bijective).
On a :
$$
\begin{array}{lll}
N_E(b(x))&=&N_E(\sum_{i=1}^nx_ie_i)\\
&\leq & \sum_{i=1}^n|x_i|N_E(e_i)\\
&\leq & n ||x||_\infty \x Sup~N_E(e_i)\\
&\leq & M ||x||_\infty
\end{array}$$
Bref, si $||x||_\infty$ est petit alors, $N_E(b(x))$ l'est aussi et si $||x-x_0||_\infty$ est petit alors, $N_E(b(x)-b(x_0)) = N_E(b(x - x_0))$\\
Donc $b$ est continue de $(\R^n, ||.||_\infty)$ dans $(E,N_E)$.\\
Soit $r>0$, donc $b(\bar{B}(0,r))$ est compacte dans $E$.\\
Donc $b(S(0,r))$ est compacte dans $E$ $(S(0,r) = \bar{B}(0,r)\backslash B(0,r))$.\\
$N_E(b(S(0,r)))$ compact de $\R^+$ et $0\notin N_E(b(S(0,r)))$.\\
Donc $0$ séparé de $N_E(b(S(0,r)))$ donc $N_E(b(S(0,r))) \geq m >0$\\
Donc si $0\neq u\in E$ $u=N_E(u)\x \frac{u}{N_E(u)}$
On a :
$$\begin{array}{lll}
N_E(b(x))&=&N_E(||x||_\infty\x b(\frac{x}{||x||_\infty}))\\
&=&||x||_\infty \x N_E(b(\frac{x}{||x||_\infty}))\\
&\geq& ||x||_\infty m\hspace{5em}||x||_\infty = \fracun{m}N_E(b(x))
\end{array}$$

\paragraph{On a montré :} 
$N_E(u) \leq M ||x||_\infty$ pour $||x||_\infty = \fracun{m}N_E(b(x))$.\\
Donc $x\overset{b}{\rightarrow} u= \sum x_ie_i$ et $x = b^{-1}(u)$ sont toutes deux continues.\\
Donc aussi "$(E,N_E)$" et  "$(\R^n,||.||_\infty)$" sont équivalentes.\\

\paragraph{Conséquence de ce lemme}
\begin{enumerate}
\item Sur $E$ de dimension finie, toutes les normes sont équivalentes :
$$(E,N_1) \leftarrow (\R^n, ||.||_\infty) \rightarrow (E,N_2)$$
(en particulier sur $\R^n$, toute norme est équivalente à $||.||_\infty$)

\item Soit $A$ un fermé borné dans $(E,N_E)$, alors $b^{-1}(A)$ (image réciproque) est fermé dans $\R^n$.\\
Mais (d'après le lemme), $N_E(A)$ bornée $\Rightarrow ||x||_\infty$  est bornée sur $b^{-1}(A)$.\\
Donc $b^{-1}(A)$ est compact dans $\R^n$, donc $b(b^{-1}(A))=A$ est compact ($b$ bijective).
\end{enumerate}


\cor
Tout \evn de dimension finie est donc localement compact : $\bar{B}(u,r) = \{u\in i, N_E(u) \leq r \}$ sont fermées bornées donc compactes.

\defi
On dit que $f$ bijective de $(E,N_E)$ dans $(F,N_F)$ est bi-continue (ou un homéomorphisme) ssi $f$ et $f^{-1}$ (application réciproque de $f$) sont continues.

\subsubsection{Théorème (Riesz)}
Un \evn $(E,N_E)$ est localement compact ssi $E$ est de dimension finie.

\dem
On a déjà démontré que la dimension finie implique localement compact.\\
Montrons l'implication inverse :\\
Soit $(E,N_E)$ est localement compact.\\
Donc $0$ a un voisinage compact dans $E$, $0\in V$ compact.\\
D'où $\exists 0\in B(0,r)\subseteq \overline{B(0,r)} = \bar{B}(0,r) \subseteq V$ ($V$ fermé).\\
$\bar{B}(0,1)$ est compacte, 
$$\begin{array}{lll}
\overline{B(0,1)} &\subseteq & \cup_{x\in \overline{B(0,1)}} B(x,\fracun{2})\\
&\subseteq & \cup_{i\in I}B(x_i, \fracun{2})
\end{array}$$
Soit $F = <x_i> \subseteq E$ (dimension finie car engendré par un nombre fini de vecteur).\\
Tout point $x\in E$ est adhérant à $F$.\\
$\frac{x}{N_E(x)} \in \bar{B}(0,1)$ d'où $\frac{x}{N_E(x)} \in B(x_i, \fracun{2})$ d'où $N_E(x-N_E(x)x_i)\leq \fracun{2}N_E(x)$





\subsection{Continuité uniforme}














\chapter{Calcul différentiel}

\chapter{Séries de Fourier}

\end{document}
