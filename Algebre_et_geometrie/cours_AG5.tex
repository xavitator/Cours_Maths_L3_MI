\documentclass[a4paper, oneside]{report}
\usepackage[top=3cm, bottom = 3cm, left = 3cm, right = 3cm]{geometry}
\usepackage{amsfonts,amsmath,amssymb}
\usepackage[utf8]{inputenc}
\usepackage[francais]{babel}
\usepackage{graphicx}
\usepackage{polynom}
\usepackage[T1]{fontenc}
\usepackage{mathenv}
\usepackage{mdwtab}
\usepackage{array}
\usepackage{pgf,tikz} % \begin{tikzpicture}
\usepackage{pdfpages} %\includepdf[page={1-5}]{truc.pdf}
\usepackage[colorlinks=true,linkcolor=black]{hyperref}
\usetikzlibrary{arrows,calc}
\usepackage[amsthm]{ntheorem} % package pour les environnements de theorem

\theorempreskip {2em}
\theorempostskip{2em}
\theoremstyle{break}
\theoremseparator{\vspace{0.5em}}
\newtheorem{thm}{Théoreme}[section] % reset theorem numbering for each chapter
\newtheorem{defi}[thm]{Définition}
\newtheorem{propr}[thm]{Propriété}
\newtheorem{propr-defi}[thm]{Propriété-Définition}
\newtheorem{propo}[thm]{Proposition}
\newtheorem{cor}[thm]{Corollaire}
\newtheorem{cons}[thm]{Conséquence}
\newtheorem{lemme}[thm]{Lemme}
\newtheorem{nota}[thm]{Notation}

\theorembodyfont{\normalfont}
\newtheorem{exem}[thm]{Exemple}
\newtheorem*{demo}{Démonstration}
\newtheorem*{rappel}{Rappel}
\newtheorem*{term}{Terminologie}
\newtheorem{remar}[thm]{Remarque}
\newtheorem{exo}[thm]{Exercice}


\newcommand{\x}{\times}
\newcommand{\R}{\mathbb{R}}
\newcommand{\Rb}{\bar{\R}}
\newcommand{\N}{\mathbb{N}}
\newcommand{\K}{\mathbb{K}}
\newcommand{\C}{\mathbb{C}}
\newcommand{\D}{\mathbb{D}}
\newcommand{\Z}{\mathbb{Z}}
\newcommand{\Q}{\mathbb{Q}}
\newcommand{\U}{\mathbb{U}}
\newcommand{\A}{\mathcal{A}}
\newcommand{\displayastyle}{\displaystyle}
\newcommand{\sev}{sous espace vectoriel }
\newcommand{\sevs}{sous espaces vectoriels }
\newcommand{\ev}{espace vectoriel }
\newcommand{\mdg}{morphisme de groupes }
\newcommand{\mdgs}{morphismes de groupes }
\newcommand{\aut}{automorphisme }
\newcommand{\auts}{automorphismes }
\newcommand{\isom}{isomorphisme }
\newcommand{\isoms}{isomorphismes }
\newcommand{\pcsd}{produit de cycles à support disjoints }
\newcommand{\sg}{sous-groupe }
\newcommand{\sgs}{sous-groupes }
\newcommand{\fong}{\overset{\sim}{\rightarrow}}
\newcommand{\fracun}[1]{\frac{1}{#1}}
\newcommand{\inj}{\hookrightarrow}
\newcommand{\surj}{\twoheadrightarrow}

\begin{document}

\title{Cours d'Algèbre et géométrie I}
\date{11/09/2018}
\author{Bernard Keller}
\maketitle

\tableofcontents{}
\chapter{Groupes}

\section{Motivation}

\subsection{Éléments de symétrie}
\begin{itemize}
\item 3 symétries orthogonales : $\sigma_A$, $\sigma_B$, $\sigma_C$
\item rotation $\rho$ d'angle $\frac{2\pi}{3}$
\item rotation $\rho^2$ d'angle $\frac{4\pi}{3}$
\item l'identité
\end{itemize}

$D_3=\{Id, \sigma_A, \sigma_B, \sigma_C, \rho, \rho^2\}$ est un groupe diédral.\\
On peut composer les éléments de l'ensemble $D_3$ et on restera dans $D_3$.\\
La composition est associative.\\
Elle admet un élément neutre, l'identité.\\
Chaque élément admet un inverse.

\begin{figure}[h]
	\begin{center}
		\begin{tikzpicture}
		\coordinate (A) at (-2,0);
		\coordinate (B) at (2,0);
		\coordinate (X) at (120:4);
		\coordinate (C) at ($(B) + (X)$);
		\draw (A) -- (B) -- (C) -- cycle;
		\coordinate (AB) at (barycentric cs:A=1,B=1);
		\coordinate (AC) at (barycentric cs:A=1,C=1);
		\coordinate (BC) at (barycentric cs:B=1,C=1);
		\draw[dashed] (A) -- ($(BC)$) node[near start, below right] {$\sigma_A$};
		\draw[dashed] (B) -- ($(AC)$) node[near start, below left] {$\sigma_B$};
		\draw[dashed] (C) -- ($(AB)$) node[near start, below right] {$\sigma_C$};
		\node (Abis) at (A) {};
		\node (Bbis) at (B) {};
		\node (Cbis) at (C) {};
		\draw[->,>=latex] (Cbis) to[bend left] node[midway, above right]{$\rho$} (Bbis);
		\draw[->,>=latex] (Cbis) to[bend right] node[midway, above left]{$\rho^2$} (Abis);
		\end{tikzpicture}
		\caption{Le groupe diédral $D_3$}
	\end{center}
\end{figure}

\section{Définition et premiers exemples}

\begin{defi}
	Un groupe est un couple $(G,*)$, où $G$ est un ensemble et :
	$$*:G\x G \rightarrow G, (g,h)\mapsto g*h$$
	est son appellation telle que :
	\begin{enumerate}
		\item $*$ est associative, c'est à dire :
		$$(x*y)*z=x*(y*z)$$
		\item $*$ admet un élément neutre e, c'est à dire :
		$$e*x = x = x*e$$
		\item tout élément $x\in G$ admet un inverse x', c'est à dire :
		$$x*x'=e=x'*x$$
	\end{enumerate}	
\end{defi}

\begin{remar}
	\begin{enumerate}
		\item Souvent, on écrit $xy$ au lieu de $x*y$\\
		\item L'élément neutre e est unique : en effet, si e' est un deuxième élément neutre, on a :
		$$ e = e'e = e'$$
		\item L'inverse est unique : en effet, soit x'' un deuxième inverse. On a :
		$$x''=ex''=(x'x)x''=x'(xx'')=x'e=x'$$
		On note désormais $x^{-1}$ l'inverse de $x$.\\
		\item Pour tous $x,y\in G$, on a $(xy)^{-1}=y^{-1}x^{-1}$.\\
		En effet, on a :
		$$(xy)(y^{-1}x^{-1})=(x(yy^{-1}))x^{-1} = (xe)x^{-1}=e$$
		$$(y^{-1}x^{-1})(xy)=y^{-1}(x^{-1}(xy))=y^{-1}(ey)$$
	\end{enumerate}	
\end{remar}

\begin{defi}
	Un groupe G est abélien ou commutatif si xy=yx, pour tous $x,y\in G$.	
\end{defi}

\begin{remar}
	Souvent, on notre $+$ la loi de groupe d'un groupe abélien. On note alors 0, l'élément neutre et -x l'élément inverse de $x\in G$.	
\end{remar}

\begin{exem}
	\begin{enumerate}
		\item $D_3=\{Id, \sigma_A, \sigma_B, \sigma_C, \rho, \rho^2\}$ n'est pas commutatif car $\sigma_C \circ \sigma_A = \rho$ et $\sigma_A \circ \sigma_C = \rho^{-1} = \rho ^2 \neq \rho$.
		\item $(\Z, +)$ est un groupe abélien.
		\item $(\Q,+), (\R,+), (\C,+)$ sont des groupes abéliens.
		\item $\Q^*=\Q \backslash \{0\}$ est un groupe abélien pour la multiplication. De même pour $\R^*$ et $\C^*$.
		\item Si E est un \ev sur $\R$ ou $\C$, alors $(E,+)$ est un groupe abélien.
		\item Soit $n\geq 1$ un entier, alors l'ensemble $GL_n(\R)$ des matrices inversibles $n\x n$ est un groupe pour la multiplication des matrices. Il est abélien ssi n=1.\\
		De même pour $GL_n(\Q)$ et $GL_n(\C)$.\\
		\item Soit X un ensemble (fini ou infini). Le groupe symétrique $\sigma_X$ est formé des bijections $f:X\rightarrow X$. Sa multiplication est la composition des applications. Son élément neutre est $Id_X$. L'inverse d'une bijection $f:X\rightarrow X$ est la bijection réciproque $f^{-1}:X\rightarrow X$. En particulier, pour $n\geq 1$, on a le groupe symétrique :
		$$\sigma_n = \sigma_{\{1,2,...,n\}} = \text{groupe de permutations de } \{1,...,n\}$$
		Notons que $|\sigma_n|=n!$.
	\end{enumerate}	
\end{exem}

\begin{nota}
	Soit G un groupe. Soient $g\in G$ et $n\in \N$.\\
	On note $g^n$, l'élément de G défini par récurrence :
	$$\begin{array}{lll}
	g^0&=&e\\
	g^{n+1}&=&g^ng~,~\forall n \geq 0
	\end{array}$$
	Si $n>0$, on pose $g^{-n}=(g^n)^{-1}$.	
\end{nota}

\begin{lemme}
	Soient G un groupe et $m,n\in\Z$. On a $g^{m+n}=g^mg^n$ et $(g^n)^{-1}=g^{-n}$.
\end{lemme}
	

\begin{demo}
	Il faut distinguer des cas. Les détails sont laissés en exercice.	
\end{demo}

\begin{lemme}
	Soient G et H deux groupes.\\
	Posons :
	$$K=G\x H=\{(g,h)|g\in G, h\in H \}$$
	Alors K est un groupe pour la loi :
	$$K\x K \rightarrow K,~((g,h),(g',h'))\mapsto (gg',hh')$$
\end{lemme}

\begin{demo}
	Clairement la loi est associative. Elle admet $e_K=(e_G,e_H)$ pour élément neutre et l'inverse de $(g,h)$ est $(g^{-1}, h^{-1})$, $\forall g\in G, h\in H$.
\end{demo}

\begin{defi}
	$G\x H$ muni de cette loi est le groupe produit de G par H.
\end{defi}

\begin{exo}
	$G\x H$ est abélien ssi G et H sont abéliens.
\end{exo}

\section{Sous-groupe}

\begin{defi}	
	Soit G un groupe. Un sous-groupe de G est une partie de $H\subseteq G$ telle que :
	\begin{enumerate}
	\item $e_G \in H$
	\item $\forall h,h' \in H$, on a $hh' \in H$
	\item $\forall h\in H$, on a $h^{-1}\in H$
	\end{enumerate}
\end{defi}

\begin{nota}
	On note $H\leq G$ lorsque H est un sous-groupe de G.	
\end{nota}

\begin{lemme}
	Une partie $H\subseteq G$ est un sous-groupe ssi $H\neq \emptyset$ et pour tous $h_1,h_2\in H$, on a $h_1h_2^{-1}\in H$.	
\end{lemme}

\begin{demo}
	"$\Rightarrow$" $H\neq \emptyset$ car $e_G \in H$. Si $h_1,h_2\in H$ alors $h_2^{-1}\in H$ (c) et donc $h_1h_2^{-1}\in H$ (b).\\
	"$\Leftarrow$" Comme H est non vide, on peut choisir un $h\in H$. Alors $hh^{-1}=e_G \in H$. Soient $h_1,h_2 \in H$. On a $h_2^{-1}=eh_2^{-1}\in H$. Donc $h_1h_2=h_1(h_2^{-1})^{-1}\in H$	
\end{demo}

\begin{remar}
	\begin{enumerate}
		\item Soit H un sous-groupe de G. Alors la loi de G induit une application $H\x H \mapsto H$, $(h_1,h_2)\mapsto h_1h_2$ (bien définie par b)). Muni de cette loi, H devient un groupe d'élément neutre $e_H=e_G$. Désormais tout sous-groupe d'un groupe est considéré comme un groupe de cette façon.
		\item Si $H\leq G$ et $K\leq H$, alors $K\leq G$
	\end{enumerate}
\end{remar}

\begin{exem}
	Soit G un groupe.
	
	\begin{enumerate}
		\item $\{e\} \leq G$
		\item $G\leq G$
		\item Posons $Z(G)=\{g\in G | hg=gh, \forall h\in G \}$
		Clairement, on a $e\in Z(G)$. On montre ensuite que la multiplication de deux éléments de $Z(G)$ est toujours dans $Z(G)$.\\
		Enfin, on montre que soient $g\in Z(G)$ et $h\in G$, on a $hg^{-1}=g^{-1}h$, donc $g^{-1}\in Z(G)$.\\
		Par conséquent, $Z(G)$ est un sous-groupe de G.
	\end{enumerate}
\end{exem}

\begin{defi}
	On appelle $Z(G)$ le centre de $G$.
\end{defi}

\begin{exem}
	$Z(GL_n(\R)) = \R^* \cdot I_n$
\end{exem}

\begin{exem}[de sous-groupes (suite)]
	Soit $n\geq 1$. Les parties suivantes sont des sous-groupes de $GL_n(\R)$ :
	\begin{itemize}
		\item $SL_n(\R)=\{A\in GL_n(\R) | det A=1 \}$
		\item $O_n(\R) = \{A\in GL_n(\R) | A^tA=I_n \}$
		\item $SO_n(\R) = SL_n(\R) \cap O_n(\R)$
	\end{itemize}
\end{exem}

\begin{nota}
	$\U = \{z\in \C | |z|=1 \}$\\
	$\U_n = \{z\in \C | z^n =1 \}$ où $n\geq 1$\\
	$\U_n$ = {racines n-ièmes de 1}\\
	Ce sont des sous-groupes de $\C^*$\\
\end{nota}

\begin{remar}
	On a $\U_n \leq \U \leq \C^*$ et $\U_n \leq \U_{mn}$ $\forall n,m\geq 1$.\\
\end{remar}

\begin{nota}
	Pour $n\in \Z$, on pose :
	$$n\Z = \{nk| k\in \Z \}$$
\end{nota}

\begin{thm}
	\begin{enumerate}
		\item $n\Z \leq \Z$
		\item Soit H un sous-groupe de $\Z$. Il existe un et un seul $n\in \N$ tq $H=n\Z$.\\
		Si $H\neq \{0\}$, alors n est le plus petit entier strictement positif contenu dans H.
	\end{enumerate}
\end{thm}

\begin{demo}
	\begin{enumerate}
		\item est clair
		\item Soit $H \leq \Z$. Si $H=\{0\}$, alors $H=0.\Z$. Supposons donc que $H\neq \{0\}$.\\
		Soit $0\neq x\in H$. Alors $-x\in H$. Donc H contient au moins un entier strictement positif. Soit $E=\{x\in H | x>0 \}$. Alors E est une partie non vide de $\N$.\\
		Donc il existe dans E un plus petit élément n. Comme $n\in H$, on a $n\Z \subseteq H$.\\
		Montrons que $n\Z \supseteq H$. Soit $x\in H$. Supposons $x>0$, alors $x\in E$ et $x\geq n$.\\
		La division euclidienne de x par n s'écrit $x=n.q+r$, où $q,r\in \Z$ et $0\leq r\leq n$.\\
		Comme x et nq sont dans H, r est dans H.\\
		Or on a $0\leq r<n$ et n était le plus petit entier positif contenu dans H. Donc r=0 et $x=nq \in n\Z$.\\
		Donc $H=n\Z$. Finalement, si m,n sont des entiers positifs et $m\Z=n\Z$, alors $m=n$.
	\end{enumerate}
\end{demo}

\section{Sous-groupe engendré par une partie}
Soit $G$ un groupe.

\begin{lemme}
	Si $(G_i)_{i\in I}$ est une famille de sous-groupes, alors $\cap_{i\in I}G_i$ est encore un sous-groupe.
\end{lemme}

\begin{demo}
	Exercice facile.
\end{demo}

\begin{defi}
	Soit $S$ une partie de $G$. Si $S=\emptyset$, on pose $<S>=\{e\}$.\\
	Si $S\neq \emptyset$, on pose :
	$$<S> = \cap_{H~sous-groupe~tq~H\supseteq S}H$$
	On appelle $<S>$ le sous-groupe engendré par $S$.
\end{defi}

\begin{remar}
	$<S>$ est le plus petit des sous-groupe contenant $S$.
\end{remar}

\begin{defi}
	$S\subseteq G$ est une partie génératrice si $<S>=G$.\\
	$G$ est monogène s'il admet un singleton comme partie génératrice.\\
	$G$ est cyclique s'il est monogène et fini.
\end{defi}

\begin{exem}
	$(\Z,+)$ est monogène (engendré par $S=\{1\}$) et infini.\\
	$\U_n$, $n\geq 1$, est monogène et fini, donc cyclique.
\end{exem}

\begin{lemme}
	Soit S une partie non vide de G. On a :
	$$<S> = \{g_1g_2...g_n | n\in \N, g_i\in S~ou~g_i^{-1}\in S~pour~tout~i  \}$$
	
\end{lemme}

\begin{demo}
	Notons H le membre de droite. Clairement, H est un sous-groupe et contient S. Donc $H \supseteq <S>$.\\
	Soit K un autre sous-groupe contenant S. Alors pour tout $s\in S$, on a $s\in K$ et $s^{-1}\in K$.\\
	Comme K est stable par produit, K contient H donc H est le plus petit sous-groupe de G contenant S, cad $H=<S>$.
\end{demo}

\section{Morphismes de groupes}

\begin{defi}
	Soient $G$ et $H$ deux groupes. Un morphisme de groupes (appelé aussi homomorphisme) est une application $f:G\rightarrow H$ tq $f(xy)=f(x)f(y)$ $\forall x,y\in G$	
\end{defi}

\begin{remar}
	Dans ce cas, on a automatiquement $f(e_H)=e_H$ et $f(x^{-1})= f(x)^{-1},~\forall x\in G$\\
	En effet, on a :\\
	$f(e)=f(ee)=f(e)f(e)$. En multipliant à gauche par $f(e)^{-1}$, on trouve $e=f(e)$\\
	$f(x^{-1})f(x)=f(x^{-1}x)=f(e)=e$. En multipliant à droite par $f(x)^{-1}$, on trouve $f(x^{-1})=f(x)^{-1}$
\end{remar}

\begin{exem}
	\begin{enumerate}
		\item $x\mapsto exp(x)$ est un morphisme de groupe de $(\R,+)$ vers $(\R^*, \cdot)$
		\item $x\mapsto ln(x)$ est un morphisme de groupe de $(\R^*, \cdot)$ vers $(\R,+)$
		\item $det~:~GL_n(\R)\rightarrow \R^*$ est un morphisme de groupes. De même pour $GL_n(\C)$ et $GL_n(\Q)$
		\item Soient $E$ et $F$ deux espaces vectoriels sur $\R$, soit $f:E\rightarrow F$ une application linéaire.\\
		Alors en particulier, $f$ est un morphisme de groupes de $(E,+)$ vers $(F,+)$.
		\item Soient $G$ un groupe et $H\leq G$ un sous-groupe. Alors l'inclusion $H\mapsto G$ est un morphisme de groupe
	\end{enumerate}
\end{exem}

\begin{thm}
	Soit $G$ un groupe. Pour tout $g\in G$, il existe un unique \mdg $f:(\Z,+)\rightarrow G$ tel que $f(1)=g$.
\end{thm}

\begin{demo}
	Pour l'existence, posons $f(n)=g^n, n\in \Z$, alors $f(1)=g^1=g$ et $f(m+n)=g^{n+m}=g^ng^m=f(n)f(m)$ pour tous $n,m\in \Z$\\
	Pour l'unicité, notons que si $n>1$, on a $f(n)= f(1+...+1)=f(1)...f(1)=g...g=g^n$\\
	On doit aussi avoir $f(0)=e$ et $f(-n)=f(n)^{-1}=g^{-n}$ pur tout $n>0$.	
\end{demo}

\begin{thm}
	Soient $G$ un groupe et $n\geq 1$. Pour tout $g\in G$ tq $g^n=e$, il existe un unique \mdg $f:\U_n \rightarrow G$ tq $f(c)=g$, où $c=e^{\frac{2\pi i}{n}}$
\end{thm}

\begin{demo}
	On a $\U_n = \{1,c,...,c^{n-1}\}$.\\
	Montrons l'unicité. On doit avoir :
	$$f(c^k)=f(c)^k=g^k\hspace{5em}\forall 0\leq k \leq n-1$$
	Pour montrer l'existence, définissons $f$ par cette formule. Vérifions que $f$ est un morphisme.\\
	Soient $0\leq k\leq n-1$. Soit $k+l=qn+r$, la division euclidienne de $k+l$ par $n$. On a :
	$$f(c^kc^l)= f(c^{k+l})=f(c^r)=g^r$$
	$$f(c^k)f(c^l)=g^kg^l=g^{k+l}=g^r$$
\end{demo}

\begin{lemme}
	\begin{enumerate}
		\item La composée de deux \mdgs est un \mdg.
		\item Si $f:G\rightarrow H$ est un \mdg et $f$ est bijectif, alors l'application réciproque $f^{-1}:H\rightarrow G$ est encore un \mdg.
	\end{enumerate}
\end{lemme}

\begin{demo}
	\begin{enumerate}
		\item Soient $G\overset{\psi}{\rightarrow} H \overset{\varphi}{\rightarrow} K$ des \mdgs. Pour $x,y\in G$, on a :
		$$\varphi\psi (x,y)=\varphi (\psi (xy))=\varphi (\psi(x)\psi(y)) = \varphi(\psi (x))\varphi(\psi(y))=\varphi\circ \psi(x)\cdot \varphi \circ \psi(y)$$
		\item Soient $x,y\in H$. Il s'agit de monter que :
		$$f^{-1}(xy)=f^{-1}(x)f^{-1}(y)$$
		Comme $f$ est injective, il suffit de monter que les images par $f$ des deux cotés sont égales.\\
		En effet, on a :
		$$f(f^{-1}(xy))=xy~et~f(f^{-1}(x)f^{-1}(y))=xy$$
	\end{enumerate}
\end{demo}

\begin{defi}
	Un isomorphisme est un \mdg bijectif. Deux groupes $G$ et $H$ sont isomorphes s'il existe un isomorphisme $f:G\rightarrow H$.\\
	On écrit alors $G\cong H$, et on écrit une flèche $\fong$ pour désigner un isomorphisme.
\end{defi}

\begin{exem}
	\begin{enumerate}
		\item On a des isomorphismes inverses l'un de l'autre ($exp$ et $ln$)
		\item Soit $\sigma \in O_2$ tq $\sigma(1)=2$ et $\sigma(2)=1$. On a un isomorphisme :
		$$\begin{array}{lll}
		(\{\pm 1\}, \cdot)&\fong & O_2\\
		1&\mapsto &Id\\
		-1&\mapsto&\sigma
		\end{array}$$
		
		\item Soit $D_3$ le groupe des symétries d'un triangle équilatéral ($D_3=\{Id, \sigma_A, \sigma_B, \sigma_C, \rho, \rho^2\}$), on a : 
		$$f:D_3 \fong O_3$$
		en envoyant chaque élément de symétrie $g$ sur la permutation des sommets $f(g)$ qu'il induit.
	\end{enumerate}
\end{exem}

\begin{defi}
	Soit $G$ un groupe. Un automorphisme de $G$ est un isomorphisme $f:G\rightarrow G$.\\
	On note $Aut(G)$ l'ensemble des automorphismes de $G$. C'est un sous-groupe du groupe symétrique $O_G$ de l'ensemble $G$.
\end{defi}

\begin{exem}
	Pour tout $g\in G$, on a l'application de conjugaison par $g$ :
	$$cg : G \rightarrow G,~x\mapsto gxg^{-1}$$
	C'est un \mdg  car $cg(xy)=cg(x)cg(y)$\\
	C'est bijectif : sa réciproque est $cg^{-1}$ car $cg^{-1}(cg(x))=x$ $\forall x\in G$ et $cg(cg^{-1}(x)) = x$ $ \forall x\in G$\\
	Donc $cg$ est un automorphisme de $G$ appelé l'automorphisme intérieur associé à $g$
\end{exem}

\begin{propr}
	\begin{enumerate}
		\item L'application $G\rightarrow Aut(G)$, $g\mapsto cg$ est un \mdg
		\item L'ensemble des \auts intérieurs est un sous-groupe de $Aut(G)$
	\end{enumerate}
\end{propr}

\begin{demo}
	En exercice.
\end{demo}

Soient $G$ et $H$ deux groupes et $f:G\rightarrow H$ un morphisme.

\begin{defi}
	Le noyau de $f$ est :
	$$Ker(f)=\{g\in G | f(g) =e \} \subseteq G$$
	L'image de $f$ est :
	$$Im(f)=\{f(g)| g\in G\} \subseteq H$$
\end{defi}

\begin{thm}
	\begin{enumerate}
		\item $Ker(f)\leq G$
		\item $Ker(f)=\{e\}$ ssi $f$ est injective
		\item $Im(f)\leq H$
		\item $Im(f)=H$ ssi $f$ est surjective
	\end{enumerate}
\end{thm}

\begin{demo}
	\begin{enumerate}
		\item On a $e\in Ker(f)$ car $f(e)=e$. Soient $x,y\in Ker(f)$, alors :
		$$f(xy^{-1})=f(x)f(y)^{-1}=e.e^{-1}$$
		Donc $xy^{-1}\in Ker(f)$
		
		\item Supposons $f$ injective. Alors $f(g)=e=f(e)$ implique $g=e$. Donc $Ker(f)=\{e\}$.\\
		Réciproquement, supposons que $Ker(f)=\{e\}$. Soient $x,y\in G$ tq $f(x)=f(y)$.\\
		Alors $f(xy^{-1})=f(x)f(y)^{-1}=e$. Donc $xy^{-1}\in Ker(f)=\{ e \}$.\\
		Donc $xy^{-1}=e$ et $x=y$.
		
		\item On a $e=f(e)\in Im(f)$. Soient $f(x),f(y)\in Im(f)$. Alors :
		$$f(x)f(y)^{-1}=f(x)f(y^{-1})=f(xy^{-1})\in Im(f)$$
		
		\item est clair.
	\end{enumerate}	
\end{demo}

\begin{thm}
	\begin{enumerate}
		\item Soit $G'$ un sous-groupe de $G$. Alors $f(G')$ est un sous-groupe de $Im(f)$
		\item Soit $H'$ un sous-groupe de $H$. Alors $f^{-1}(H')$ est un sous-groupe de $G$ contenant $Ker(f)$
		\item Les applications $G'\mapsto f(G')$ et $H'\mapsto f^{-1}(H')$ sont des bijections inverses l'une de l'autre entre l'ensemble des sous-groupes de $G$ contenant $Ker(f)$ et l'ensemble des sous-groupes de $Im(f)$
	\end{enumerate}
\end{thm} 

\begin{demo}
	\begin{enumerate}
		\item On a $e=f(e)\in f(G')$. Si $x,y\in G'$ et donc $f(x),f(y)\in f(G')$, alors :
		$$f(x)f(y)^{-1}=f(xy^{-1})\in f(G')$$
		
		\item On a $f(e)=e \in H'$ donc $e\in f^{-1}(H')$.\\
		Soient $x,y\in f^{-1}(H')$, alors :
		$$f(xy^{-1})=f(x)f(y)^{-1} \in H'$$
		Donc $xy^{-1}\in H'$.
		
		\item Soit $G'\leq G$ un sous-groupe contenant $Ker(f)$, alors clairement $G'\subseteq f^{-1}(f(G'))$\\
		Réciproquement, soit $x\in f^{-1}(f(G'))$. Alors $f(x)\in f(G')$. Soit $y\in G'$ tq $f(x)=f(y)$.\\
		Alors $y^{-1}x\in Ker(f)\subseteq G'$. Donc :
		$$x = y.y^{-1}x \in G'$$
		Soit $H'$ un sous-groupe de $Im(f)$. Alors clairement $H'\supseteq f(f^{-1}(H'))$.\\
		Réciproquement, soit $f(g)\in H'$. Alors $g\in f^{-1}(H')$ et $f(g)\in f(f^{-1}(H'))$.
	\end{enumerate}
\end{demo}

\section{Ordre d'un élément}
Soit $G$ un groupe.\\

\begin{defi}
	L'ordre de $G$ est le cardinal $|G|$ de l'ensemble $G$.
\end{defi}

\begin{exem}
	\begin{enumerate}
		\item L'ordre de $(\Z,+)$ est infini
		\item L'ordre de $\U_n$ est $n$
	\end{enumerate}
\end{exem}

\begin{nota}
	Pour $g\in G$, on pose $<g> := <\{ g \}>$.
\end{nota}

\begin{propr}
	Soit $g\in G$. On suppose qu'il existe $n\geq 1$ tq $g^n=e$.
	\begin{enumerate}
		\item On a $<g> = \{g^i | 0\leq i \leq n-1 \}$. En particulier, l'ordre de $<g>$ est $\leq n$
		\item Si on note $d$ l'ordre de $<g>$, alors :
		$$d=min\{t\geq 1 | g^t=e \}$$
	\end{enumerate}
\end{propr}

\begin{demo}
	\begin{enumerate}
		\item $"\supseteq"$ est clair. Réciproquement, on sait que tout élément de $<g>$ est de la forme $g^i$ pour un $i \in \Z$.\\
		Soit $i=qn+r$ la division euclidienne de $i$ par $n$. Alors on a :
		$$g^i=g^{qn+r}=g^r \in \{g^k | 0\leq k \leq n-1 \}$$
		
		\item Posons $s=min\{t \geq 1 | g^t=e \}$. Alors par 1), on a :
		$$<g> = \{g^i | 0\leq i \leq s-1 \}$$
		Pour $0\leq i < j \leq s-1$, les puissances $g^i$ et $g^j$ sont distinctes. Sinon, on aurait $g^{j-i}=e$ mais $j-i<s$. Donc $s=|<g>| = d$.
	\end{enumerate}	
\end{demo}

\begin{defi}
	Soit $g\in G$. Si $<g>$ est infini, l'ordre de $g$ est infini.\\
	Si $<g>$ est fini, l'ordre de $g$ est le plus petit entier $d\geq 1$ tq $g^d=e$	
\end{defi}

\begin{remar}
	\begin{enumerate}
		\item Donc on a que l'ordre de $g$ est égale à l'ordre de $<g>$
		\item Si $d<\infty$ est l'ordre de $G$, alors :
		$$d\Z=\{n\in \Z | g^n =e \}$$
		\item Etant donné $t\geq 1$, l'élément $g$ est d'ordre $t$ ssi $g^t=e$ et $g^{t'}\neq e$ pour tout diviseur strict $t'$ de $t$.
	\end{enumerate}	
\end{remar}

\begin{exem}
	Soient $n\geq 1$ et $k\in \Z$. Soit $c=e^{\frac{2\pi i}{n}}\in \U$.\\
	Alors $c^k\in \U_n$ est d'ordre $\displayastyle\frac{ppcm(n,k)}{k}$	
\end{exem}

\begin{thm}[Théorème de Lagrange]
	Soit $G$ un groupe fini. Alors, l'ordre de tout sous-groupe $G'\leq G$ divise l'ordre de $G$.
\end{thm}

\begin{cor}
	Soit $G$ un groupe fini, alors tout élément $g\in G$ est d'ordre fini et son ordre divise l'ordre de $G$.
\end{cor}

\begin{cons}
	Soit $G$ un groupe fini dont l'ordre est un nombre premier.\\
	Alors tout sous-groupe de $G$ est égal à $G$ ou à $\{e\}$.\\
	En particulier, si $e\neq g\in G$, alors $G=<g>$. Donc $G$ est cyclique.
\end{cons}

\begin{demo}
	Soit $H$ un sous-groupe de $G$.
	
	Pour $g\in G$, on pose :
	$$gH=\{gh | h\in H \}$$
	alors $|gH|=|H|$, $\forall g\in G$, car on a les bijections réciproques l'une de l'autre.
	
	Montrons que pour tous $g,g'\in G$, on a :
	$$gH\cap g'H \neq 0 \Rightarrow gH=g'H$$
	En effet, si on a $gh=g'h'$, pour $h,h'\in H$, alors pour $h''\in H$, on a :
	$$gh''=g'g'^{-1}gh''=g'h'h^{-1}h'' \in g'H$$
	Donc $gH\subseteq g'H$ et de même $g'H \subseteq gH$. Donc $gH=g'H$.
	
	Notons que la réunion des $gH$, $g\in G$, est $G$ car $g=g \cdot e \in gH$, pour $g\in G$. Il s'ensuit que $\{gH| g\in G \}$ est une partition de $G$.
	
	Chaque $gH$ a le même nombre d'éléments : $|H|$
	
	Donc $|G| = |H| \cdot |\{gH | g\in G \}|$.
\end{demo}

\section{Les treillis des sous-groupes}

\begin{defi}
	Soit $X$ un ensemble. Une relation $R$ sur $X$ est un sous-ensemble $R\subseteq X\x X$.
	
	On note $xRy$ ("$x$ est en relation avec $y$") lorsque $(x,y)\in R$.	
\end{defi}

\begin{defi} 
	Une relation $R$ est une relation d'ordre ssi :
	\begin{itemize}
	\item (réflexivité) $\forall x \in X$, $xRx$
	\item (antisymétrique) $\forall x,y\in X$ $(xRy~et~yRx)\Rightarrow x=y$
	\item (transitive) $\forall x,y,z\in X$ $(xRy~et~yRz) \Rightarrow xRz$
	\end{itemize}
\end{defi}

\begin{defi}
	Un ensemble $(X,R)$ muni d'une relation d'ordre s'appelle un ensemble ordonné.
\end{defi}

\begin{exem}
	\begin{enumerate}
		\item $(\R,\leq )$ est un ensemble ordonné.
		\item Soit $n\geq 1$, $X$ l'ensemble des diviseurs positifs de $n$, avec $R$ la relation $xRy \Leftrightarrow x~divise~y$, est un ensemble ordonné.\\
		\item $X$ un ensemble, $P(x)$ l'ensemble des parties de $X$ avec $ARB\Rightarrow A\subseteq B$
	\end{enumerate}	
\end{exem}

\begin{defi}
	Soit $(X,R)$ un ensemble ordonné et soit $A\subseteq X$ un ensemble. Un minorant (resp majorant) de A est un $x\in X$ tq $xRa$ $\forall a \in A$ (resp $aRx,~\forall a\in A$), le plus petit (resp le plus grand) élément de A est un minorant (resp un majorant) qui est dans A.
\end{defi}

Dorénavant notons $\leq $ toute relation d'ordre sur un ensemble $X$.

\begin{defi}
	Un treillis est un ensemble ordonné $(X, \leq )$ tq $\forall (x,y)\in X\x X$ il existe dans $X$ un plus petit majorant $Sup(x,y)$ de $\{x,y\}$ et un plus grand minorant $Inf(x,y)$ de $\{x,y\}$.
\end{defi}

\begin{exem}
	\begin{enumerate}
		\item $(\R , \leq)$ est un treillis (évident)
		\item Soit $n\geq 1$ un entier, $X=\{d\in \N | d~divise~n \}$ muni de $x\leq y \Leftrightarrow x|y$ est un treillis pour $sup(k,l)=ppcm(k,l)$ (qui est encore un diviseur de $n$) et $inf(k,l)=pgcd(k,l)$
		\item $X$ un ensemble, $P(x)$ l'ensemble des parties de $X$. $(P(x),\subseteq)$ est un treillis avec $A,B\in P(x)$ $sup(A,B)=A\cup B$ et $inf(A,B)=A\cap B$
		\item $V$ un $K$-\ev, $K$ un corps ($\R,\C,...$), $Gr(V)$ l'ensemble des sous $K$-\ev de $V$ est un treillis pour $\subseteq$ car :\\
		$\forall U,W\in Gr(V)$ $sup(U,W)=\{u+w\in V |u\in U,w\in W \}$ est le plus petit \sev de $V$ qui contient $U$ et $W$, et $inf(U,W)=U\cap W$ est le plus grand \sev de $V$ inclus dans $U$ et dans $V$.\\
		\item $G$ un groupe, $L(G)$ l'ensemble des sous-groupes de $G$ est un treillis pour $\subseteq$ car :\\
		$H,H'\in L(G)$ $sup(H,H')=<H,H'>$ (groupe engendré par $H$ et $H'$), et $inf(H,H')=H\cap H'$
	\end{enumerate}
\end{exem}

\begin{defi}
	Soit $(X,\leq )$ un treillis. Son diagramme de Hasse est le graphe orienté où :
	\begin{itemize}
		\item les sommets sont les éléments $x\in X$
		\item on met une flèche $x\rightarrow y$ si $y$ est minimal parmi les éléments $\geq x$ distincts.
	\end{itemize}
\end{defi}

\begin{exem}
	\begin{enumerate}
		\item $(P(\{1,2\},\subseteq))$ :\\
		\begin{center}
			\begin{tikzpicture}
			\node (12) at (0,2) {$\{1,2\}$};
			\node (1) at (-1,1) {$\{1\}$};
			\node (2) at (1,1) {$\{2\}$};
			\node (0) at (0,0) {$\emptyset$};
			\draw[->,>=latex] (0) to (1);
			\draw[->,>=latex] (0) to (2);	
			\draw[->,>=latex] (1) to (12);
			\draw[->,>=latex] (2) to (12);
			\end{tikzpicture}
		\end{center}
		
		\item $(X=\{\text{ensemble des diviseurs de 20}\}, |)$, on a $X=\{1,2,4,5,10,20\}$ :\\
		\begin{center}
			\begin{tikzpicture}
			\node (1) at (0,0) {$1$};
			\node (2) at (1,1) {$2$};
			\node (4) at (2,2) {$4$};
			\node (5) at (-1,1) {$5$};
			\node (10) at (0,2) {$10$};
			\node (20) at (1,3) {$20$};
			\draw[->,>=latex] (1) to (2);
			\draw[->,>=latex] (1) to (5);	
			\draw[->,>=latex] (2) to (4);
			\draw[->,>=latex] (2) to (10);
			\draw[->,>=latex] (5) to (10);
			\draw[->,>=latex] (4) to (20);
			\draw[->,>=latex] (10) to (20);
			\end{tikzpicture}
		\end{center}
	\end{enumerate}
\end{exem}

\begin{lemme}
	Soit $n\geq 1$ un entier, $\zeta = e^{\frac{2i\pi}{n}}\in \U_n$.\\
	Les sous-groupes de $\U_n$ sont exactement les $\zeta^{d\Z}$ avec $d|n$. De plus $\zeta^{d\Z} \subseteq \zeta^{d'\Z} \Leftrightarrow d'|d$
\end{lemme}

\begin{demo}
	Soit $e : \begin{array}{lll}
	\Z &\rightarrow &\U_n\\
	k&\mapsto& \zeta^k
	\end{array}$.\\
	C'est un morphisme de groupes. Donc $\forall H$ sous-groupe de $\U_n$, $e^{-1}(H)=\{k\in \Z |e(k)\in H \}$ est un sous-groupe de $\Z$.\\
	De plus $e^{-1}(H)\supseteq e^{-1}(1)$ (car $1\in H$).\\
	On connaît les sous-groupes de $\Z$ : les $d\Z$.\\
	$\exists d\in \N~tq~ e^{-1}(H)=d\Z$ donc $e^{-1}(1)=\{k\in \Z | \zeta^k= e^{\frac{2i\pi k}{n}} =1 \}=n\Z$\\
	Donc $n\in n\Z \subseteq d\Z \Rightarrow d|n$. Donc $H=\zeta^{d\Z}$ avec $d|n$.
	$$\zeta^{d\Z}\subseteq \zeta^{d'\Z}\Leftrightarrow d\Z \subseteq d'\Z \Leftrightarrow d'|d$$
\end{demo}

\begin{exem}
	\begin{enumerate}
		\item Treillis de $\U_{20}$ :\\
		D'après le lemme précédent, les sous-groupes de $\U_{20}$ sont $<\zeta^{20}>$, $<\zeta^{10}>$, $<\zeta^{5}>$, $<\zeta^{4}>$, $<\zeta^{2}>$ et $<\zeta^{1}>$ :
		
		\begin{figure}[h]
			\begin{center}
				\begin{tikzpicture}
				\node (20) at (0,0) {$<\zeta^{20}>$};
				\node (10) at (1,1) {$<\zeta^{10}>$};
				\node (5) at (2,2) {$<\zeta^{5}>$};
				\node (4) at (-1,1) {$<\zeta^{4}>$};
				\node (2) at (0,2) {$<\zeta^{2}>$};
				\node (1) at (1,3) {$<\zeta^{1}>$};
				\draw[->,>=latex] (20) to (4);
				\draw[->,>=latex] (20) to (10);	
				\draw[->,>=latex] (4) to (2);
				\draw[->,>=latex] (10) to (2);
				\draw[->,>=latex] (10) to (5);
				\draw[->,>=latex] (2) to (1);
				\draw[->,>=latex] (5) to (1);
				\end{tikzpicture}
				\caption{Diagramme de Hasse du groupe $\U_{20}$}
			\end{center}
		\end{figure}
		
		\item Treillis des sous-groupes de $D_3=\{Id, \sigma_A, \sigma_B, \sigma_C, \rho, \rho^2\}$.\\
		Soit $H$ un sous-groupe de $D_3$, $|H|\in \{1,2,3,6 \}$, donc on a :
		\begin{itemize}
			\item $|H| = 1\Rightarrow H=\{Id\}$
			\item $|H|=2\Rightarrow H=<\sigma_A>, <\sigma_B>, <\sigma_C>$
			\item $|H|=3 \Rightarrow H= <\rho>, <\rho^2>$
		\end{itemize}
		\begin{figure}[h]
			\begin{center}
				\begin{tikzpicture}
				\node (ID) at (0,0) {$\{Id\}$};
				\node (1) at (-2.25,1) {$<\tau_1>$};
				\node (2) at (-0.75,1) {$<\tau_2>$};
				\node (3) at (0.75,1) {$<\tau_3>$};
				\node (R) at (2.25,2) {$<\rho>$};
				\node (D3) at (0,3) {$<D_3>$};
				
				\draw[->,>=latex] (ID) to (1);
				\draw[->,>=latex] (ID) to (2);
				\draw[->,>=latex] (ID) to (3);
				\draw[->,>=latex] (ID) to[bend right] (R);
				\draw[->,>=latex] (1) to (D3);
				\draw[->,>=latex] (2) to (D3);
				\draw[->,>=latex] (3) to (D3);
				\draw[->,>=latex] (R) to (D3);
				\end{tikzpicture}
				\caption{Diagramme de Hasse du groupe diédral 3}
			\end{center}	
		\end{figure}
		
		
		\item Treillis des sous-groupes de $D_4$, on pose :
		\begin{itemize}
			\item $\tau_i$ la réflexion par rapport à $\Delta_i$
			\item $\rho$ la rotation d'angle $\frac{2\pi}{4}$
		\end{itemize}
		On a $D_4=\{Id, \rho,\rho^2,\rho^3,\tau_1,\tau_2, \tau_3, \tau_4 \}$.\\
		Soit $H$ un sous-groupe de $D_4$, $|H|\in \{1,2,4,8 \}$, donc on a :
		\begin{itemize}
			\item $|H| = 1 \Rightarrow H = \{Id\}$
			\item $|H| = 2 \Rightarrow H = <\rho^2>,<\tau_1>,<\tau_2>,<\tau_3>,<\tau_4>$
			\item $|H| = 4 \Rightarrow H = <\rho>, <\tau_1,\tau_3>,<\tau_2,\tau_4>$
		\end{itemize}
		\begin{figure}[h]
			\begin{center}
				\begin{tikzpicture}[scale=0.7]
				\node (E) at (8,-3) {$\{e\}$};
				\node (1) at (4.5,-1) {$<\tau_1>$};
				\node (3) at (6.25,-1) {$<\tau_3>$};
				\node (R2) at (8,-1) {$<\rho^2>$};
				\node (2) at (9.75,-1) {$<\tau_1>$};
				\node (4) at (11.5,-1) {$<\tau_4>$};
				\node (R) at (8,1) {$<\rho>$};
				\node (13) at (6.25,1) {$<\tau_1,\tau_3>$};
				\node (24) at (9.75,1) {$<\tau_2,\tau_4>$};
				\node (D4) at (8,3) {$D_4$};
				
				\draw[->,>=latex] (E) to (1);
				\draw[->,>=latex] (E) to (3);
				\draw[->,>=latex] (E) to (R2);
				\draw[->,>=latex] (E) to (2);
				\draw[->,>=latex] (E) to (4);
				\draw[->,>=latex] (1) to (13);
				\draw[->,>=latex] (3) to (13);
				\draw[->,>=latex] (R2) to (13);
				\draw[->,>=latex] (R2) to (R);
				\draw[->,>=latex] (R2) to (24);
				\draw[->,>=latex] (2) to (24);
				\draw[->,>=latex] (4) to (24);
				\draw[->,>=latex] (13) to (D4);
				\draw[->,>=latex] (R) to (D4);
				\draw[->,>=latex] (24) to (D4);
				\end{tikzpicture}
			\end{center}
			\caption{Diagramme de Hasse du groupe diédral 4}	
		\end{figure}
		
	\end{enumerate}	
\end{exem}

\chapter{Actions de groupes}

\section{Relations d'équivalence}

\begin{defi}
	$X$ un ensemble. Une relation $R$ sur $X$ est une relation d'équivalence ssi :
	\begin{enumerate}
		\item (réflexivité) $\forall x\in X$ $xRx$
		\item (symétrie) $\forall x,y\in X$ $xRy\Leftrightarrow yRx$
		\item (transitivité) $\forall x,y,z\in X$ $(xRy~et~yRz)\Rightarrow xRz$
	\end{enumerate}
\end{defi}

\begin{exem}
	Soit $G$ un groupe.
	\begin{enumerate}
		\item $H\subseteq G$ un sous-groupe. On définit $g,g'\in G$ $g\sim_Hg'$ si $\exists h\in H | g'=gh$. C'est une relation d'équivalence.
		\item $\forall g,g'\in G$, on définit $g\sim g'$ si $\exists x\in G | g'=xgx^{-1}$. C'est une relation d'équivalence.
		\item $X=L(G)$ l'ensemble des sous-groupes de G avec la relation $H\sim H'$ si $\exists g\in G | H'=gHg^{-1}$
	\end{enumerate}
\end{exem}

\begin{defi}
	Soit $(X,R)$ un ensemble muni d'une relation d'équivalence.
	
	La classe d'équivalence de $x\in X$ est $\bar{x}=\{y\in X | xRy \}$.
	
	Le quotient de $X$ par $R$ est $X/R=\{\bar{x} | x\in X \}$.
	
	L'application $\begin{array}{lll}
	X&\rightarrow& X/R\\
	x&\mapsto&\bar{x}\\
	\end{array}$ s'appelle la surjection canonique.
\end{defi}

\begin{exem}
	Dans le cas 1) de l'exemple précédent, $g\in G$, $\bar{g}=\{gh|h\in H \}=gH$ et on note $G/\sim_H=G/H$.\\
	$G/H$ n'est pas un groupe en général.
\end{exem}

\begin{propr}
	\begin{enumerate}
		\item $X/R$ est une partition de $X$
		\item $\forall x,y\in X$ $xRy\Leftrightarrow \bar{x}=\bar{y}$
	\end{enumerate}
\end{propr}

\begin{demo}
	\begin{enumerate}
		\item Soit $\bar{x},\bar{y}\in X/R$. Supposons $\bar{x}\cap \bar{y}\neq \emptyset$ et montrons que $\bar{x}=\bar{y}$.\\
		$\exists z\in \bar{x}~et~z\in \bar{y}$.\\
		Montrons que $\bar{x}\subseteq \bar{y}$ :\\
		Soit $z'\in \bar{x}$, $z'Rz$ et $zRz'$ $\Rightarrow$ $z'Ry$ $\Rightarrow$ $z'\in \bar{y}$.\\
		On montre que $\bar{y}\subseteq \bar{x}$ par un raisonnement identique.\\
		Cela montre que les classes d'équivalences sont disjointes ou confondues.\\
		Et $\forall x\in X$ $x\in \bar{x}$. Donc les classes d'équivalences forment une partition de $X$.\\
		
		\item "$\Rightarrow$" Supposons $xRy$, soit $z\in \bar{x}$, on a $ zRy$ $z\in \bar{y}$.\\
		Donc de même, on a $\bar{y}\subseteq \bar{x}$. Donc $\bar{x}=\bar{y}$\\
		"$\Leftarrow$" Supposons $\bar{x}=\bar{y}$, $y\in \bar{y}=\bar{x}$, $y\in \bar{x}$, donc $yRx$.
	\end{enumerate}
\end{demo}

\begin{thm}
	Soit $(X,R)$ un ensemble avec une relation d'équivalence.\\
	Soit $\pi$ la surjection canonique.\\
	Soit $f$ une application de $X$ dans $Y$. Les assertions suivantes sont équivalentes :
	\begin{enumerate}
		\item ($\forall x,y \in R~xRy\Rightarrow f(x)=f(y)$)
		\item ($\exists ! \bar{f}: X/R \rightarrow Y$ telle que $f=\bar{f}\circ \pi$)
	\end{enumerate}
\end{thm}

\begin{demo}
	Supposons 1).
	
	\begin{itemize}
		\item unicité de $\bar{f}$ :\\
		Si $\bar{f}_1$ et $\bar{f}_2$ vérifient $\bar{f}_1\circ \pi = f = \bar{f}_2 \circ \pi$.
		
		Soit $\bar{x}\in X/R$ $\bar{x}=\pi(x)$, et on a :
		$$\bar{f}_1(\bar{x})=(\bar{f}_1\circ \pi)(x)=f=(\bar{f}_2\circ \pi)(x)=\bar{f}_2(\bar{x})$$
		
		\item Existence de $\bar{f}$ :
		
		Soit $\chi\in X/R,~\exists x\in X | \pi(x)=\bar{x}=\chi$.
		
		On pose $\bar{f}(\chi)=f(x)$. Cette définition est indépendante du choix de $x$, car :
		
		si $y\in X$ vérifie $\pi(y)=\chi\Rightarrow \pi(x)=\pi(y)$
		
		D'après le lemme , on a $xRy \Rightarrow f(x)=f(y)$. Donc cette définition définit une application $\bar{f}:X/R \rightarrow Y$ et elle vérifie $f=\bar{f}\circ \pi$ par construction.
	\end{itemize}
	
	Supposons 2). Soit $x,y\in X, xRy \Rightarrow \pi(x)=\pi(y)\Rightarrow (\bar{f}\circ \pi)(x)=(\bar{f}\circ \pi)(y)\Rightarrow f(x)=f(y)$
	
\end{demo}

\begin{remar}
	Lorsque $f$ vérifie 1) du théorème, on dit que $f$ passe au quotient par $R$ et que $\bar{f}$ est induite par $f$.
\end{remar}

\begin{lemme}
	Soit $(X,R)$, $f$ vérifiant les assertions du théorème précédent :
	\begin{enumerate}
		\item $\bar{f}$ est surjective $\Leftrightarrow$ $f$ l'est aussi
		\item $\bar{f}$ est injective $\Leftrightarrow$ ($\forall x,y\in X,~f(x)=f(y)\Rightarrow xRy$)
	\end{enumerate}
\end{lemme}

\begin{demo}
	\begin{enumerate}
		
		\item Supposons $\bar{f}$ surjective, $f=\bar{f}\circ \pi$ est surjective, car $\bar{f}$ et $\pi$ sont surjectives.
		
		Supposons $f$ surjective, soit $y\in Y$, $\exists x \in X | f(x)=\bar{f}(\pi(x))=y$, $\bar{f}$ est surjective.
		\item à faire en exercice
	\end{enumerate}
\end{demo}

\begin{propr}
	Soit $n\geq 1$ entier, soit $e:\begin{array}{lll}
	\Z &\rightarrow &\U_n\\
	k&\mapsto&e^{\frac{2\pi i k}{n}}
	\end{array}$.
	
	Soit $R$ la relation d'équivalence $xRy\Leftrightarrow n|x-y$.
	
	Notons $\Z/n\Z=\Z/R$, alors $e$ induit une bijection $\bar{e}:\Z/n\Z \rightarrow \U_n$ avec $e=\bar{e}\circ \pi$
\end{propr}

\begin{demo}
	L'existence de $\bar{e}$ découle du théorème et de $xRy \Leftrightarrow n|x-y \Leftrightarrow \exists k\in \Z | x=y+nk$.\\
	Ceci implique que $e(x)=e(y)$.\\
	La surjection de $\bar{e}$ découle du lemme et de la surjectivité de $e$.\\
	L'injection de $\bar{e}$ découle de $\forall x,y\in \Z$ $e(x)=e(y)\Leftrightarrow xRy$, et du lemme.
\end{demo}

\section{Définition d'une action de groupe}

\begin{defi} 
	Soient $X$ un ensemble, $G$ un groupe. Une action de $G$ sur $X$ est une application $\begin{array}{lll}
	G\x X &\rightarrow & X\\
	(g,x)&\rightarrow & g.x\\
	\end{array}$ telle que :
	\begin{enumerate}
	\item $\forall x\in X$ $e.x=x$
	\item $\forall g,h\in G$ $\forall x\in X$ $(gh).x=g.(h.x)$
	\end{enumerate}
\end{defi}

\begin{defi}
	Un $G$-ensemble est un ensemble muni d'une action du groupe $G$.	
\end{defi}


\begin{exem}
	\begin{enumerate}
		\item Le groupe diédral $D_3=\{Id, \sigma_A, \sigma_B, \sigma_C, \rho, \rho^2\}$ agit sur l'ensemble $\{1,2,3\}$ des sommets du triangle équilatéral.
		\item Le groupe symétrique $\sigma_n$ agit sur l'ensemble $X=\{1,\cdots,n\}$ par $\sigma.x := \sigma(x), ~\forall \sigma \in \sigma_n,~\forall x\in X$.\\
		
		Soit $G$ un groupe.
		\item Soit $H\subseteq G$ un sous-groupe. Alors $H$ agit sur $G$ par :
		$$H\x G \rightarrow G,~ (h,g)\mapsto hg$$
		On appelle cette action, l'action de $H$ sur $G$, par transition à gauche.
		\item L'application $G\x G \rightarrow G,~ (g,x) \mapsto gxg^{-1}$ est une action de $G$ sur lui-même (en effet, on a $e.x = e.x.e^{-1} =x$, $\forall x\in G$ et $(gh).x=ghx(gh)^{-1}g(hxh^{-1})g^{-1}=g(hx),~\forall g,h\in G, \forall x\in G$).\\
		On l'appelle l'action de conjugaison de $G$ sur lui-même.
		\item Soit $X$ l'ensemble des sous-groupes de $G$. L'application :
		$$G\x X \rightarrow X,~(g,K)\mapsto gKg^{-1}$$
		est une action de groupe. On l'appelle l'action de conjugaison de $G$ sur l'ensemble de ses sous-groupes.
		\item Soit $n\geq 1$. L'application :
		$$GL_n(\R)\x \R^n \rightarrow \R^n,~(g,v)\mapsto g(v)$$
		est une action de groupe.
		\item Soit $n\geq 1$. L'application :
		$$\begin{array}{lll}
		GL_n(\R)\x M_n(\R)&\rightarrow & M_n(\R)\\
		(P,M)&\mapsto & PMP^{-1}
		\end{array}$$
		
		est une action de groupe de $GL_n(\R)$ sur $M_n(\R)$.
	\end{enumerate}
\end{exem}

\begin{propr}
	Soient $G$ un groupe et $X$ un ensemble.
	\begin{enumerate}
		\item Une action de $G$ sur $X$ : pour tout $g\in G$, soit $\varphi_g :X \rightarrow X$ l'application $x\mapsto gx$, alors $\varphi_g \in \sigma_X$ et l'application $G\rightarrow \sigma_X~g\mapsto \varphi_g$ est un morphisme de groupes.
		\item Soit $f:G\rightarrow \sigma_X$ est un morphisme de groupes. Alors il existe une unique action de $G$ sur $X$ tq :
		$$g.x = (f(g))(x),~~\forall g\in G,\forall x\in X$$
		Comme le montre la proposition, on a une bijection naturelle entre l'ensemble des actions de $G$ sur $X$ et l'ensemble des morphismes de groupes de $G$ vers $\sigma_X$
	\end{enumerate}
\end{propr}

\begin{demo}
	\begin{enumerate}
		\item Pour tout $g\in G$, l'application $\varphi_g$ est bijective de réciproque $\varphi_{g^{-1}}$ car :
		$$\varphi_g\varphi_{g^{-1}}(x)=\varphi_g(g^{-1}x)=g(g^{-1}x)=(gg^{-1})x=ex=x$$
		et de la même façon :
		$$\varphi_{g^{-1}}\varphi_g(x)=ex=x$$
		On a pour $g,h\in G$ :
		$$\varphi_{gh}(x)=(gh)(x)=g(hx)=\varphi_g\circ \varphi_h(x),~~\forall x\in X$$
		Donc l'application $g\mapsto \varphi_g$ est bien un morphisme de groupe $G\rightarrow \sigma_X$
		\item On définit l'application $G\x X \rightarrow X$ par $g.x = (f(g))(x),~~\forall g\in G,\forall x\in X$, vérifions qu'il s'agit d'une action.\\
		$$ex=(f(e))x=Id_X(x)=x,~~\forall x\in X$$
		et 
		$$g(hx)=f(g(hx))=f(g)(f(h)(x))=(f(g)\circ f(h) )(x)=f(gh)(x)=gh.x$$
		pour tous $g,h\in G$ et tout $x\in X$
	\end{enumerate}
\end{demo}


\begin{defi}
	Soient $G$ un groupe et $X$ un ensemble. Une action à droite de $G$ sur $X$ est une application :
	$$X\x G\rightarrow X,~(x,g)\mapsto x.g$$
	telle que :
	\begin{enumerate}
		\item $x.e=x,~\forall x\in X$
		\item $x.(gh)=(xg).h,~\forall g,h\in G,~\forall x\in X$
	\end{enumerate}
\end{defi}

\begin{exem}
	Soit $n\geq 1$, alors l'application :
	$$\begin{array}{lll}
	M_n(\R)\x GL_n(\R)&\rightarrow & M_n(\R)\\
	(P,M)&\mapsto & MP
	\end{array}$$
	est une action à droite de $GL_n(\R)$ sur $M_n(\R)$.
\end{exem}

\begin{remar}
	Soit $X$ un ensemble muni d'une action à droite d'un groupe $G$. On définit $g.x := x.g^{-1}$ $\forall g\in G$.\\
	C'est une action à gauche de $G$ sur $X$ car :
	$$ex=xe^{-1}=xe=x$$
	et 
	$$(gh)x=x.(gh)^{-1}=x(h^{-1}g^{-1})=(xh^{-1})g^{-1}=(hx)g^{-1}=g(hx)$$
	$\forall x\in X, ~\forall g,h\in G$.\\
	On obtient ainsi une bijection entre les actions à droite de $G$ sur $X$ et les actions à gauche de $G$ sur $X$.
\end{remar}

\section{Orbites et stabilisateurs}
Soient $G$ un groupe et $X$ un $G$-ensemble.

\begin{defi}
	Pour $x\in X$, l'orbite de $x$ est :
	$$G.x=\{gx | g\in G \}$$
	Le stabilisateur de $x$ est :
	$$Stab_G(x)=G_x=\{g\in G | gx=x \}$$
\end{defi}

\begin{figure} [h]
	\begin{center}
		\begin{tikzpicture}
		\draw (0,0) arc (160:75:6) node[midway, circle, fill, inner sep = 0pt, minimum size = 0.7em, label=above left:$x$] (X) {}  node[near end, above] {$g \cdot x$};
		\draw[->, >=latex] (X) to [out=240,in=340,looseness=15] 
		node[below right] {$h$ : stabilisateur de $x$} (X); 
		\end{tikzpicture}
		\caption{Orbite de $x$ sous $G$}
	\end{center}	
\end{figure}

\begin{remar}
	\begin{enumerate}
		\item Soit $x\in X$. L'orbite $G.x$ contient $x=e.x$. Le stabilisateur $G_x$ est un sous-groupe car $e.x=x$, et $g(hx)=gx=x,~\forall g,h\in G_x$, et si $g\in G_x$ alors $g^{-1}\in G_x$ car $g^{-1}x = x \Leftrightarrow g(g^{-1}x)=gx \Leftrightarrow ex = x$.
		\item On définit de façon analogue les orbites et stabilisateurs d'une action à droite.
	\end{enumerate}
\end{remar}

\begin{exem}
	Soit $H\subseteq G$ un sous-groupe.
	\begin{enumerate}
		\item Pour l'action $H\x G \rightarrow G,~(h,g)\mapsto hg$, l'orbite d'un $g\in G$ est $Hg$, la classe à gauche modulo $H$ de $g$.\\
		Le stabilisateur de $g\in G$ est formé des $h\in H$ tq $hg=g \Leftrightarrow h=e$. Donc $Stab_H(g)=\{e\}$.\\
		\item Pour l'action à droite :
		$$G\x H \rightarrow G,~(g,h)\mapsto gh$$
		l'orbite de $g\in G$ est la classe à droite $gH$. En outre, $Stab_H(g)=\{e\}$
	\end{enumerate}
\end{exem}

\begin{propr-defi}
	Soit $\sim$ la relation sur $X$ tq :
	$$x\sim y \Leftrightarrow y\in G \cdot x$$
	Alors $\sim$ est une relation d'équivalence sur $X$ appelée la relation d'équivalence associée à l'action de $G$ sur $X$.
\end{propr-defi}

\begin{demo}
	On vérifie que $\sim$ est :
	\begin{itemize}
		\item réflexive : $x\sim x$ car $x=ex$
		\item symétrique : $x\sim y$ $\sim$ $y\sim x$ car $y=gx\Leftrightarrow g^{-1}y=x$
		\item transitive : Si $x\sim y$ et $y\sim z$, alors $x\sim z$ car si $y=gx$ et $z=hy$, alors $z=hy=h(gx)=(hg)x$
	\end{itemize}
\end{demo}

\begin{remar}
	Pour tout $x\in X$, la classe d'équivalence de $x$ est égale à l'orbite $G \cdot x$.	
\end{remar}

\begin{defi}
	Le quotient de $X$ par $G$ est l'ensemble $G\backslash X := X/\sim$ formé des orbites de $G$ dans $X$.
\end{defi}

\begin{remar}
	On définit de façon analogue la relation d'équivalence et l'ensemble quotient d'une action à droite de $G$ sur $X$.\\
	L'ensemble quotient est alors noté $X/G$.
\end{remar}

\begin{propr}
	L'ensemble des orbites est une partition de $X$.
\end{propr}

\begin{demo}
	En effet, ce sont des classes d'équivalence pour une relation d'équivalence.
\end{demo}

\begin{defi}
	Soit $X$ un $G$-ensemble non-vide.
	
	L'action de $G$ sur $X$ est :
	\begin{itemize}
		\item transitive s'il n'y a qu'une seule orbite
		\item fidèle si $\forall g\in G$, on a :
		$$gx=x,~\forall x\in X \Rightarrow g=e$$
		\item libre si tous les stabilisateurs sont triviaux ($Stab_G(x)=\{e\},~\forall x\in X$).
	\end{itemize}
\end{defi}

\begin{remar}
	\begin{enumerate}
		\item On définit de façon analogue les notions correspondantes pour les actions à droite.
		\item L'action de $G$ sur $X$ est transitive ssi $G\backslash X$ est un singleton.
		\item L'action de $G$ sur $X$ est fidèle ssi le morphisme associé $G\rightarrow \sigma_X$ a pour noyau $\{e\}$, c'est à dire ssi $G\rightarrow \sigma_X$ est injectif.
	\end{enumerate}	
\end{remar}

\begin{exem}
	\begin{enumerate}
		\item Pour tout sous-groupe $H$ de $G$, l'action de $H$ sur $G$ par translations à gauche (ou à droite) est libre (car $Stab_H(g)=\{h\in H |hg=g \}=\{e\}$), donc fidèle.\\
		Elle est transitive ssi $H=G$ (s'il n'y a qu'une seule orbite, elle est égale à $G$, donc $G$ est l'orbite sous $H$ de $e$ mais cette orbite est $H.e=H$)
		\item Soit $n\geq 1$. Considérons l'action :
		$$GL_n(\R)\x \R^n \rightarrow \R^n,~(g,v)\mapsto gv $$
		L'orbite d'un vecteur $v\neq 0$ est $\R^n\backslash \{0\}$ (en effet si $v_1,...v_n$ est une base tq $v_1 = v$ et $w_1,...,w_n$ est une base tq $w_1=w\neq 0$, il existe un unique $g\in GL_n(\R)$ tq $g(v_i)=w_i$, $\forall i$, en particulier $gv=w$).\\
		L'orbite de $v=0$ est $\{0\}$.\\
		Il y a donc exactement 2 orbites : $\R^n\backslash \{0\}$ et $\{0\}$.\\
		Donc l'action n'est pas transitive.\\
		Elle est fidèle (car si $gv=v~\forall v$, alors $ge_i=e_i$, pour $i\in [|1,n|]$ et $g=Id$).\\
		Elle n'est pas libre car $Stab_{GL_n(\R)}(0)=GL_n(\R)$
		
		\item L'action $GL_n(\R)\x \R^n* \rightarrow \R^n*,~(g,v)\mapsto gv $ est transitive, fidèle et non libre. En effet, $Stab_{GL_n(\R)}(e_1)=[e_1,*,*,...,*]$ avec $*$ des vecteurs quelconques.
		
		\item Soit $C=\{(x_1,x_2,x_3) \in \R | \forall i\text{ on a } x_i= \pm 1 \}$ l'ensemble des sommets d'un cube de $\R^3$ centré en l'origine. Soit $G = \{g \in O_3(\R) | g(C)=C \}$ ($O_n$ est l'ensemble des matrices orthogonales de taille $n\x n$).\\
		L'action :
		$$G\x C \rightarrow C, (g,x)\mapsto gx$$
		est transitive (combiner des rotations et des symétries).\\
		Elle est fidèle (les vecteurs $\left[\begin{array}{l}
		1\\
		1\\
		1
		\end{array}\right]$, $\left[\begin{array}{l}
		1\\
		1\\
		-1
		\end{array}\right]$, $\left[\begin{array}{l}
		1\\
		-1\\
		1
		\end{array}\right]$ forment une base de $\R^3$).\\
		Elle n'est pas libre (la rotation d'angle $\frac{2\pi}{3}$ et d'axe $\R \left[\begin{array}{l}
		1\\
		1\\
		1
		\end{array}\right]$ est dans G et dans le stabilisateur de $\left[\begin{array}{l}
		1\\
		1\\
		1
		\end{array}\right]$).
	\end{enumerate}
\end{exem}

\section{Aspects numériques}

Soit $G$ un groupe et soit $X$ un $G$-ensemble (ensemble muni d'une action de $G$).

\begin{thm}
	Soit $x\in X$. Soit $\pi : G \rightarrow G/Stab_G(x)$ la projection canonique.
	
	Il existe une et une seule application :
	$$\varphi : G/Stab_G(x) \rightarrow G.x$$
	telle que $\varphi \circ \pi(g)=gx$ pour tout $g\in G$.
	
	Cette application est bijective.
\end{thm}

\begin{demo}
	Soit $H=Stab_G(x)$. Comme $\pi$ est surjective, l'application $\varphi$, si elle existe, est unique.
	
	Soient $g\in G$ et $h\in H$. On a :
	$$(gh)x = g(hx)=gx\hspace{5em}h\in Stab_G(x)$$
	Donc l'application $\tilde{\varphi}: G\rightarrow G.x$ vérifie $\tilde{\varphi}(gh)=\tilde{\varphi}(g), \forall h\in H, \forall g\in G$.
	
	Donc $\tilde{\varphi}(g)$ ne dépend que de la classe $gH\in G/H$. Par passage au quotient par $H$, $\tilde{\varphi}:G \rightarrow G.x$ induit $\varphi :G/H \rightarrow G.x$. Clairement, $\varphi$ est surjective.
	
	Supposons que $g_1, g_2\in G$ sont tels que $\varphi(g_1)=\varphi(g_2)$. Alors $g_1x=g_2x$, donc $x=g_1^{-1}g_2x$ et $g_1^{-1}g_2\in H$ et $g_2\in g_1H$. Donc on a $g_2H=g_1H$, ou $\pi(g_1)=\pi(g_2)$.
	
	Cela montre que $\varphi$ est injective.
\end{demo}

\begin{remar}
	L'ensemble $G/Stab_G(x)$ est un $G$-ensemble pour l'action naturelle :
	$$ g.\pi(g') := \pi(gg'),\hspace{5em}\forall g,g'\in G$$
	où $\pi : G\rightarrow G/Stab_G(x)$ est la projection canonique. La bijection canonique $G/Stab_G(x)\fong G.x$ est en fait un isomorphisme de $G$-ensembles.\\
	En particulier, tout $G$-ensemble transitif est isomorphe à un $G$-ensemble de la forme $G/H$ pour un sous-groupe $H$ de $G$.
\end{remar}

\begin{cor}
	On suppose $G$ et $X$ finis.
	\begin{enumerate}
		\item Pour tout $x\in X$, on a $|G.x|=\frac{|G|}{|Stab_G(x)|}$. En particulier, $|G.x|$ divise $|G|$.
		\item Choisissons un élément $x_i$ dans chaque orbite, $1\leq i \leq n$. On a :
		$$|X|=\sum_{i=1}^{n}\frac{|G|}{|Stab_G(x_i)|}$$
	\end{enumerate}
\end{cor}

\begin{remar}
	Ces égalités sont appelées \textbf{équations aux classes}.
\end{remar}

\subsection{Applications}

\subsubsection{Application 1}

Soit $p$ un nombre premier. Supposons que $G$ est un $p$-groupe, c'est à dire son ordre est une puissance de $p$.

\begin{defi}
	Un élément $x$ d'un $G$-ensemble $X$ est un point fixe si $gx=x$ $\forall g\in G$.
\end{defi}

Soient $G$ un $p$-groupe, et $X$ un $G$-ensemble fini.

Si $x\in X$ n'est pas un point fixe, le cardinal de l'orbite $|G.x|$ est un diviseur $>1$ de $|G|$.

Donc $p$ divise $|G.x|$. D'où :

\begin{cor}
	Si $G$ est un $p$-groupe et $X$ un $G$-ensemble fini, alors :
	$$|X| \equiv |X^G|~mod~p$$
	où $X^G$ est l'ensemble des points fixes de $G$ dans $X$.
\end{cor}

\subsubsection{Application 2}

\begin{thm}[de Cauchy]
	Soient $G$ un groupe fini et $p$ un nombre premier qui divise $|G|$, alors $G$ contient un élément d'ordre $p$.
\end{thm}

\begin{demo}[d'après John McKay]
	Soit :
	$$X=\{(g_1,...,g_p)\in G^p | g_1g_2....g_p =e \}$$
	Notons que :
	$$g_1g_2...g_p = e$$
	$$\Rightarrow g_2...g_p=g_1^{-1}$$
	$$\Rightarrow g_2...g_pg_1=e$$
	Donc $X$ est stable par permutation cyclique des composantes. Donc le groupe cyclique $H=\U_p$ agit sur $X$ par :
	$$\zeta(g_1,g_2...g_p) := (g_2...g_pg_1)$$
	où $\zeta = e^{\frac{2\pi i}{p}}$.\\
	Les points fixes sont les $(g,...,g)\in G^p$ tq $g^p=e$. Cela veut dire que ou bien $g=e$ ou bien $g$ est un élément d'ordre $p$.\\
	Par le corollaire précédent, on a :
	$$|X^H|=|X|~mod~p$$
	Or $X$ est de cardinal $|G|^{p-1}$ (l'application $X \rightarrow G^{p-1},~(g_1,...,g_p)\mapsto (g_2,...,g_p)$ est bijective). Donc :
	$$|X^H|=0~mod~p$$
	Il existe donc au moins un point fixe autre que $(e,...,e)$.	
\end{demo}

\chapter{Groupes symétriques}

\section{Définition et premières propriétés}

\subsubsection{Rappel}
Si $E$ est un ensemble, le groupe symétrique $\sigma_E$ est le groupe des bijections $f:E\rightarrow E$ avec la composition des applications pour loi. On note :
$$\sigma_n := \sigma_{\{1,2,...,n \}} \hspace{5em} n\geq 1$$
et on l'appelle le $n$-ième groupe symétrique. Il est d'ordre $n!$.

\begin{remar}
	Si $E$ et $F$ sont deux ensembles et $\varphi:E\rightarrow F$, une bijection, on a un isomorphisme de groupes :
	$$\sigma_E \rightarrow \sigma_F,~f\mapsto \varphi\circ f \circ \varphi^{-1}$$
	En particulier, l'étude de $\sigma_E$ pour un ensemble fini de cardinal $n$ se ramène à celle de $\sigma_n$.
\end{remar}

\begin{nota}
	Si $\sigma\in \sigma_n$, on le décrit à l'aide du tableau :
	$$\begin{array}{llll}
	1&2&...&n\\
	\sigma(1)&\sigma(2)&...&\sigma(n)
	\end{array}$$
\end{nota}

\begin{remar}
	\begin{enumerate}
		\item Le groupe $\sigma_n$ agit sur $\{1,...,n\}$ par :
		$$\sigma.i=\sigma(i), \hspace{2em} \forall i\in \{1,...,n\}, \forall \sigma \in \sigma_n$$
		
		\item Cette action est fidèle et transitive
		
		\item Pour tout $i\in \{1,...,n\}$, la stabilisateur de $i$ dans $\sigma_n$ est isomorphe à $\sigma_{\{1,2,...,n \}\backslash \{i\}}$
	\end{enumerate}
\end{remar}

\begin{defi}
	Soit $\sigma \in \sigma_n$. Le support de $\sigma$ est l'ensemble :
	$$supp(\sigma)=\{i\in \{1,...,n \} | \sigma(i)\neq i \}$$
\end{defi}

\begin{propr}
	\begin{enumerate}
		\item Deux permutations à supports disjoints commutent
		\item Les groupes symétriques $\sigma_1$ et $\sigma_2$ sont abéliens. Pour $n\geq 3$, la centre de $\sigma_n$ est trivial.
	\end{enumerate}	
\end{propr}

\begin{demo}
	On peut et on va supposer $n\geq 3$.
	\begin{enumerate}
		\item Soient $\sigma_1,\sigma_2 \in \sigma_n$ tq $supp(\sigma_1)\cap supp(\sigma_2) = \emptyset$.\\
		Si l'une parmi $\sigma_1$ et $\sigma_2$ est l'identité, elles commutent bien.\\
		Supposons $supp(\sigma_1)$ et $supp(\sigma_2)$ non vides ($\sigma_i \neq Id$ $\forall i$).\\
		Soit $i\in supp(\sigma_1)$, alors $i\notin supp(\sigma_2)$ et $\sigma_1(i)\notin supp(\sigma_2)$. Donc :
		$$\sigma_1 \circ \sigma_2(i)=\sigma_1(i)$$
		$$\sigma_2 \circ \sigma_1(i)=\sigma_1(i)$$
		De même, pour $i\in supp(\sigma_2)$, on a :
		$$\sigma_1 \circ \sigma_2(i)=\sigma_2(i)$$
		$$\sigma_2 \circ \sigma_1(i)=\sigma_2(i)$$
		D'autre part, si $i\notin supp(\sigma_1)\cup supp(\sigma_2)$, alors $\sigma_1 \circ \sigma_2(i) = i = \sigma_2 \circ \sigma_1(i)$.\\
		On conclut que $\sigma_1 \circ \sigma_2(i)= \sigma_2 \circ \sigma_1(i)$
		
		\item Soit $\sigma \in \sigma_n \backslash \{Id\}$.\\
		Soient $i\in \{1,...,n\}$ tq $\sigma(i)\neq i$ et $k\in \{1,...,n\}\backslash \{i, \sigma(i) \}$.\\
		Soit $\tau$ la permutation tq :
		$$\tau(\sigma (i))=k,~\tau(k)=\sigma(i),~\tau(j)=j,~\forall j\notin \{k,\sigma(i) \}$$
		Montrons que $\tau \circ \sigma \neq \sigma \circ \tau$. En effet :
		$$\tau \circ \sigma (i) = k$$
		$$\sigma \circ \tau (i) = \sigma(i) \neq k$$
	\end{enumerate}
\end{demo}

\subsection{Transpositions et cycles}

\begin{defi}
	Soit $n\geq 2$ et soit $2 \leq l \leq n$. Soit $(a_1,...,a_l)$ une suite d'éléments 2 à 2 distincts de $\{1,...,n\}$. \\
	On note encore $(a_1,...,a_l)$ la permutation définition par :
	$$\begin{array}{ll}
	x\mapsto x & \forall x \in \{1,...,n \}\backslash \{a_1,...,a_l\}\\
	a_i\mapsto a_{i+1}& \forall 1\leq i \leq l-1\\
	a_l\mapsto a_1&
	\end{array}
	$$
\end{defi}

Une telle permutation est appelée $l$-cycle (ou cycle). Sa longueur est $l$.\\
Si $l=2$, elle est appelée la transposition de $a_1$ et $a_2$

\begin{remar}
	Soit $\sigma=(a_1,...,a_l)$ un $l$-cycle.
	\begin{enumerate}
		\item Soit $i\in \{1,...,l-1\}$, alors $\sigma^i(a_1)=a_{1+i}$. Plus généralement, on a :
		$$\sigma^i(a_j)= \left\{\begin{array}{ll}
		a_{j+i}& 1\leq j \leq l-i\\
		a_{j+i-l}& l-i+1\leq j \leq l
		\end{array}\right.$$
		Le cycle est d'ordre $l$ dans $\sigma_n$.
		
		\item Pour tout $\tau \in \sigma_n$, on a :
		$$\tau \circ (a_1,...,a_l) \circ \tau^{-1}=(\tau(a_1),...,\tau(a_l))$$
		
		\item 
		$$ (a_1,...,a_n)=(a_1,a_2)\circ...\circ (a_{l-2}, a_{l-1}) \circ (a_{l-1}, a_l)$$
		Le $l$-cycle est produit de $l-1$ transpositions.
		
		\item 
		$$(a_1,...,a_n)=(a_2,...,a_n,a_1)$$
		
		\item Soit $\tau_1$ et $\tau_2$ deux transpositions à support disjoint, alors $\tau_1\tau_2 = \tau_2\tau_1$ (qui est d'ordre 2) est appelé une \textbf{double transposition}.
		
	\end{enumerate}
\end{remar}

\begin{exem}
	\begin{enumerate}
		\item $$\sigma_2=\{e, (12)\}$$
		\item $$\sigma_3=\{e, (12), (13), (23), (123), (132)\}$$
		\item
		\begin{center}
			$\sigma_4$=\{e, (12), (13), (23), (14), (24), (34),\\
			(12)(34), (13)(24), (14)(23),\\
			(123), (132), (124),(142),(134),(143),(234),(243),\\
			(1234),(1243),(1324),(1342),(1423),(1432)\}
		\end{center} 
	\end{enumerate}
\end{exem}

\begin{thm}
	Soit $\sigma\in \sigma_n$.
	\begin{enumerate}
		\item Il existe un entier naturel $k$ et des cycles $c_1,...,c_k$ de $\sigma_n$ à supports disjoints 2 à 2 tq :
		$$\sigma = c_1...c_k$$
		
		\item Si $s$ est un entier naturel et $c_1',...,c_s'$ des cycles à supports disjoints 2 à 2 tq :
		$$\sigma = c_1'...c_s'$$
		alors $k=s$ et il existe une permutation $\tau \in \sigma_k$ tq $c_i'=c_{\tau(i)}$, $\forall 1\leq i \leq k$
	\end{enumerate}
\end{thm}

\paragraph{Idée de la démonstration :}
On fait agir le groupe $<\sigma> \subseteq \sigma_n$ sur $\{1,...,n\}$. Les orbites nous fournissent les cycles $c_i$, $1\leq i \leq k$

\begin{exem}
	$$\sigma = \left(\begin{array}{llllllllllllll}
	1&2&3&4&5&6&7&8&9&10&11&12&13&14\\
	6&7&9&4&5&10&2&3&12&13&8&11&1&14
	\end{array}\right)$$
	$\sigma = (1~6~10~13)(2~7)(3~9~12~11~8)$ est une décomposition en produit de cycles à supports disjoints 2 à 2 de $\sigma \in \sigma_{14}$.
\end{exem}

\begin{demo}
	On fait agir le sous-groupe $<\sigma>$ engendré par $\sigma$ dans $\sigma_n$ sur l'ensemble $\{1,...,n\}$.\\
	Sur cette action, l'ensemble $\{1,...,n\}$ se décompose en orbites disjointes 2 à 2. Les orbites ponctuelles sont exactement les points fixes de $\sigma$.\\
	Soient $\Omega_1,...,\Omega_r$ les orbites non ponctuelles.\\
	Le sous-groupe $<\sigma>$ permute cycliquement les éléments de chaque $\Omega_i$. Soit $a_{i1},...,a_{il_i}$ une énumération des éléments de $\Omega_i$ tq :
	$$\sigma(a_{i_j})\left\{\begin{array}{ll}
	a_{i_{j+1}}&1\leq j \leq l_i-1\\
	a_{i_1}&j=l_i
	\end{array}\right. $$
	Soit $c_i=(a_{i_1},...,a_{i_{l_i}})$, alors l'action de $c_i$ et de $\sigma$ sur l'orbite $\Omega_i$ est la même.\\
	Donc l'action de $\sigma$ et de $c_1c_2...c_r$ sur $\{1,...,n\}$ est la même.\\
	Donc $\sigma=c_1...c_r$.
\end{demo}

\subsubsection{Terminologie}
\begin{enumerate}
\item Avec les hypothèses et les notations du théorème, on dit que l'égalité $\sigma=c_1...c_r$ est la décomposition de $\sigma$ en \textbf{produit de cycles à supports disjoints}.

\item Si $\sigma$ et $\sigma '$ sont deux permutations, on dit que $\sigma$ et $\sigma '$ sont du \textbf{même type} si pour tout entier $2\leq l \leq n$, le nombre de $l$-cycles dans la décomposition de $\sigma$ en produit de cycles à support disjoints est égal au nombre de $l$-cycles dans la décomposition de $\sigma '$ en \pcsd.
\end{enumerate}

\begin{exem}
	$(12)(34)(567)$ est du même type que $(123)(45)(67)$.
\end{exem}

\begin{cor}
	Pour tout $n\geq 1$, $\sigma_n$ est engendré par l'ensemble de ses transpositions.
\end{cor}

\begin{demo}
	En effet, $\sigma_n$ est engendré par ses cycles et chaque cycle est produit de transpositions, comme on l'a vu.
\end{demo}

\begin{exo}
	Monter que $\sigma_n$ est même engendré par les $n-1$ transpositions :
	$$(1~2)(2~3)...(n-1~n)$$
\end{exo}

\begin{cor}
	Soient $n\geq 1$ et $\sigma\in \sigma_n$, alors l'ordre de $\sigma$ est le PPCM des longueurs des cycles apparaissant dans la décomposition de $\sigma$ en \pcsd.
\end{cor}

\begin{exem}
	$ord((1~2)(2~3)(4~5~6)) = PPCM(2,2,3)=6$
\end{exem}

\begin{cor}
	Soient $n\geq 1$ et $\sigma, \sigma ' \in \sigma_n$, alors on a une équivalence entre :
	\begin{enumerate}
		\item $\sigma$ et $\sigma '$ sont du même type.
		\item $\sigma$ et $\sigma '$ sont conjugués.
	\end{enumerate}
\end{cor}

\begin{demo}
	Cela provient du fait que pour un cycle $(a_1,...,a_l)$ et une permutation $\tau\in \sigma_n$, on a :
	$$\tau \circ (a_1,...,a_l) \circ \tau^{-1}=(\tau(a_1),...,\tau(a_l))$$
\end{demo}

\begin{exem}
	$\sigma = (1~2)(3~4)(5~6~7)$ et $\sigma '=(1~2~3)(4~5)(6~7)$ sont conjugués par :
	$$\sigma = \left(\begin{array}{lllllll}
	1&2&3&4&5&6&7\\
	4&5&6&7&1&2&3
	\end{array}\right)$$
\end{exem}

\begin{remar}
	Deux permutations conjuguées ont même ordre (et même signature, voir au-dessus).
\end{remar}

\section{La signature}
Soit $n\geq 1$. Le groupe symétrique $\sigma_n$ agit sur l'ensemble $\Z[X_i,...,X_n]$ des polynômes en $\Xi_1,...,X_n$ à coefficients entiers par :
$$(\sigma~P)(X_1,...,X_n):= P(X_{\sigma(1)},...,X_{\sigma(n)})$$
(clairement, $Id.P=P$ et $\sigma(\tau~P)=(\sigma~\tau)P, \forall \sigma,\tau \in \sigma_n$ et $\forall P\in \Z[X_i,...,X_n]$).\\

Soit 
$$\Delta_n := \prod_{i<j}(X_i-X_j)\in \Z[X_i,...,X_n]$$
Par exemple, on a $\Delta_2=X_1-X_2$, $\Delta_3=(X_1-X_2)(X_1-X_3)(X_2-X_3)$.\\
Toute permutation $\sigma$ envoie un $\Xi_i-X_j$ sur $X_{\sigma(i)}-X_{\sigma(j)}$ et on a $\sigma(i)<\sigma(j)$ ou $\sigma(j)<\sigma(j)$ si $i<j$.\\
Donc $\sigma$ envoie un facteur $X_i-X_j$ de $\Delta_n$ soit sur un autre facteur de $\Delta_n$ soit sur l'opposé d'un autre facteur de $\Delta_n$. Donc :
$$\sigma \Delta_n=\pm \Delta_n,~~\forall\sigma\in \sigma_n$$

\begin{defi}
	La \textbf{signature} $\epsilon(\sigma)$ de $\sigma\in \sigma_n$ est l'unique nombre $\epsilon(\sigma)\in \{1,-1\}$ tq :
	$$\epsilon(\sigma)\Delta_n=\sigma\Delta_n$$
\end{defi}

\begin{exem}
	\begin{enumerate}
		\item $\sigma = (12)\in \sigma_2$ : $\sigma(X_1-X_2)=X_2-X_1=-(X_1-X_2)\Rightarrow \epsilon(\sigma)=-1$
		\item $\sigma=(123)\in \sigma_3$ : $\sigma((X_1-X_2)(X_1-X_3)(X_2-X_3))=(X_2-X_3)(X_2-X_1)(X_3-X_1)=(-1)(-1)\Delta_3=\Delta_3\Rightarrow \epsilon(\sigma)=1$
	\end{enumerate}
\end{exem}

\begin{propr}
	\begin{enumerate}
		\item La signature est un morphisme de groupes
		$$ \epsilon : \sigma_n\rightarrow (\{1,-1\},.)$$
		
		\item Toute transposition est de signature $-1$. Tout cycle de longueur $l$ est de signature $(-1)^{l-1}$
	\end{enumerate}
\end{propr}

\begin{demo}
	\begin{enumerate}
		\item résulte du fait que $\sigma_n$ agit sur $\Z[X_i,...,X_n]$. En effet, pour $\sigma,\tau\in \sigma_n$, on a :
		$$\epsilon(\sigma\tau).\Delta_n=(\sigma\tau)\Delta_n=\epsilon(\tau)\epsilon(\sigma)\Delta_n$$
		et donc $\epsilon(\sigma\tau)=\epsilon(\sigma)\epsilon(\tau),\forall \sigma,\tau \in \sigma_n$.
		
		\item Soit $\sigma\in \sigma_n$. Une \textbf{inversion} de $\sigma$ est un couple $(u,v)$ de nombres dans $\{1,...,n\}$ tq $u<v$ mais $\sigma(u)>\sigma(v)$. Clairement, on a $\epsilon(\sigma)=(-1)^t$, où $t$ est le nombre d'inversions de $\sigma$. Soit maintenant $\sigma=(ij)$, où $1\leq i < j \leq n$.\\
		Les inversions de $\sigma$ sont :
		\begin{itemize}
			\item $(u,j)$ pour $i<u<j$
			\item $(i,u)$ pour $i<u<j$
			\item $(i,j)$
		\end{itemize}
		Le nombre des inversions de $\sigma$ est donc :
		$$2(j-i+1)+1$$
		Donc $\epsilon(\sigma)=-1$.\\
		Donc toute transposition est de signature -1.\\
		Comme un $l$-cycle $c$ est produit de $l-1$ transposition, par 1), on a $\epsilon(c)=(-1)^{l-1}$
	\end{enumerate}
\end{demo}

\begin{remar}
	Soient $n\geq 2$, et $\sigma\in \sigma_n$.
	\begin{enumerate}
		\item On dit que $\sigma$ est \textbf{paire} (respectivement \textbf{impaire}) si $\epsilon(\sigma)=1$ (respectivement $\epsilon(\sigma)=-1$)
		\item $\sigma$ est pair (resp impair) ssi $\sigma$ est produit de nombre pair (resp impair) de transpositions.
		\item Soient $l_1,...,l_r$ les cardinaux des orbites $\Omega_1,...,\Omega_r$ de $<\sigma>$ dans $\{1,...,n\}$ (y compris les orbites ponctuelles), alors on a :
		$$\epsilon(\sigma)=\epsilon(c_1...c_r)=(-1)^{l_1-1}...(-1)^{l_n-1}=(-1)^{(\sum l_i)-r}=(-1)^{n-r}$$
	\end{enumerate}
\end{remar}

\begin{defi}
	Soit $n\geq 1$. On appelle n-ième \textbf{groupe alterné} le noyau $\mathcal{A}_n$ de la signature $\epsilon:\sigma_n\rightarrow \{\pm 1\}$.
\end{defi}

\begin{remar}
	On verra que $|\mathcal{A}_n|=\frac{n!}{2}$
\end{remar}

\begin{exem}
	\begin{enumerate}
		\item $\sigma_2=\{e,(12)\}$ $\mathcal{A}=\{e\}$
		\item $\sigma_3=\{e, (12), (13), (23), (123), (132)\}$ $\mathcal{A}_3=\{e,(123),(132)\}$
		\item $|\sigma_4|=24$, on a $\mathcal{A}_4=\{e,(123),(132),(234),(243),(134),(143),(124),(142),(12)(34), (13)(24), (14)(23)\}$
	\end{enumerate}
\end{exem}

\chapter{Sous-groupes distingués, groupes quotients}
\section{Sous-groupes distingués}

Soient $G$ un groupe et $H$ un \sg de $G$.

\begin{nota}
	Pour $g\in H$, on note $gHg^{-1}=\{ghg^{-1}| h\in G\}$. C'est un \sg en tant qu'image de $H$ par l'automorphisme de conjugaison (= automorphisme intérieur)
	$$c_g:G\rightarrow G,~x\mapsto gxg^{-1}$$
\end{nota}

\begin{propr}
	Les conditions suivantes sont équivalentes :
	\begin{enumerate}
		\item $\forall g\in G$, on a $gH=Hg$
		\item $\forall g\in G$, on a $gHg^{-1}=H$
		\item $\forall g\in G$, on a $gHg^{-1}\subseteq H$
	\end{enumerate}
\end{propr}

\begin{demo}
	Clairement $1) \Leftrightarrow 2)$ et $2) \Rightarrow 3)$. Montrons que $3)\Rightarrow 2)$ :\\
	Soit $x\in G$, alors pour $g=x^{-1}$, on a :
	$$H\supseteq gHg^{-1}=x^{-1}H(x^{-1})^{-1}=x^{-1}Hx$$
	et donc $xHx^{-1}\supseteq H$.	
\end{demo}

\begin{defi}
	$H$ est \textbf{distingué} (ou \textbf{normal}) dans $G$ ssi, pour tout $g\in G$, on a $gHg^{-1}=H$. On écrit alors $H\vartriangleleft G$.
\end{defi}

\begin{defi}
Soit un groupe $G$ et $H$ un sous-groupe. Le \textbf{normalisateur} de $H$ dans $G$ est :
$$N_G(H)=\{g\in G | gHg^{-1}=H\}$$
\end{defi}

\begin{remar}
$N_G(H)$ est un \sg de $G$ contenant $H$. $H$ est distingué dans $N_G(H)$ et $N_G(H)$ est le plus grand \sg de $G$ dans lequel $H$ est distingué.\\
$H$ est distingué dans $G$ ssi $N_G(H)=G$
\end{remar}

\begin{defi}
	L'\textbf{indice} de $H$ dans $G$ est $[G:H]=|G/H| = |G|/|H|$ si $|G|$ et $|H|$ sont finis.
\end{defi}

\begin{exem}
	\begin{enumerate}
		\item $\{e\}=H \Rightarrow gHg^{-1}=\{gg^{-1}\}=\{e\}$. Donc $\{e\}\vartriangleleft G$. On a $G\vartriangleleft G$ et $Z(G)\vartriangleleft G$
		\item Si $G$ est \textbf{abélien}, alors $c_g=Id_G,~\forall g\in G$, donc $H\vartriangleleft G$ pour tout \sg $H$ de $G$.
		\item Si $H$ est \textbf{d'indice 2} dans $G$ ($|G/H|=2$) alors $G=H\cup gH=H\cup Hg$ et $gH=Hg$, $H$ est distingué dans $G$.
	\end{enumerate}
\end{exem}

\begin{lemme}
	Si $f:G\rightarrow K$ est un morphisme de groupes, alors $Ker(f)$ est distingué dans $G$.
\end{lemme}

\begin{demo}
	Soient $x\in Ker(f)$ et $g\in G$. On a:
	$$f(gxg^{-1})=f(g)f(x)f(g)^{-1}=f(g)ef(g)^{-1}=e$$
	Donc $g~Ker(f)~g^{-1}\subseteq Ker(f),\forall g\in G$ et $Ker(f)\vartriangleleft G$.
\end{demo}

\begin{remar}
	$Im(f)\subseteq K$ n'est pas distingué en général. Par exemple, si $f:G\rightarrow K$ est l'inclusion d'un \sg non distingué, alors $Im(f)=G\subseteq K$ n'est pas distingué.
\end{remar}

\begin{exem}
	\begin{enumerate}
		\item $\A_n\vartriangleleft \sigma_n$ car $\A_n=Ker(\epsilon)$
		\item $SL_n(\R)\vartriangleleft GL_n(\R)$ car $SL_n(\R)=Ker(det)$
		\item $SO_n(\R)\vartriangleleft O_n(\R)$, $SU_n(\C)\vartriangleleft U_n(\C)$ par la même raison.
		\item Quels sont les \sgs distingués de $D_4=\{e, \tau_1, \tau_2, \tau_3,\tau_4, \rho, \rho^2, \rho^3\}$
		
		\begin{center}
			\begin{tikzpicture}[scale=0.75]
			\draw (-2,-2) -- (2,-2) -- (2,2) -- (-2,2) -- cycle;
			\draw[dashed] (-3,-3) -- (3,3) node[below right] {$\tau_2$};
			\draw[dashed] (3,-3) -- (-3,3) node[below left] {$\tau_4$};
			\draw[dashed] (-3,0) -- (3,0) node[below right] {$\tau_1$};
			\draw[dashed] (0,-3) -- (0,3) node[below right] {$\tau_3$};
			\draw[->,>=latex] (1,0) arc (0:90:1) node[near end, above right] {$\rho$};	
			
			\node (E) at (8,-3) {$\{e\}$};
			\node (1) at (4.5,-1) {$<\tau_1>$};
			\node (3) at (6.25,-1) {$<\tau_3>$};
			\node (R2) at (8,-1) {$<\rho^2>$};
			\node (2) at (9.75,-1) {$<\tau_1>$};
			\node (4) at (11.5,-1) {$<\tau_4>$};
			\node (R) at (8,1) {$<\rho>$};
			\node (13) at (6.25,1) {$<\tau_1,\tau_3>$};
			\node (24) at (9.75,1) {$<\tau_2,\tau_4>$};
			\node (D4) at (8,3) {$D_4$};
			
			\draw[->,>=latex] (E) -- (1);
			\draw[->,>=latex] (E) -- (3);
			\draw[->,>=latex] (E) -- (R2);
			\draw[->,>=latex] (E) -- (2);
			\draw[->,>=latex] (E) -- (4);
			\draw[->,>=latex] (1) -- (13);
			\draw[->,>=latex] (3) -- (13);
			\draw[->,>=latex] (R2) -- (13);
			\draw[->,>=latex] (R2) -- (R);
			\draw[->,>=latex] (R2) -- (24);
			\draw[->,>=latex] (2) -- (24);
			\draw[->,>=latex] (4) -- (24);
			\draw[->,>=latex] (13) -- (D4);
			\draw[->,>=latex] (R) -- (D4);
			\draw[->,>=latex] (24) -- (D4);
			\end{tikzpicture}
		\end{center}	
	\end{enumerate}
	
	Notons $\sigma_D$ la symétrie orthogonale par rapport à une droite $D$ et $f$ une isométrie. Alors :
	$$f\circ \sigma_D \circ f^{-1}=\sigma_{f(D)}$$
	Donc $D_4$, $\{e\}$, $<\tau_1,\tau_3>$, $<\tau_2,\tau_4>$, $<\rho>$, $<\rho^2>$ sont distingués et $<\tau>$, $<\tau_3>$, $<\tau_2$ et $<\tau_4>$ ne sont pas distingués.\\
	On a $N_{D_4}(<\tau_1>)=<\tau_1,\tau_3>$, $N_{D_4}(<\tau_2>)=<\tau_2,\tau_4>$.
\end{exem}

\begin{remar}
\begin{enumerate}
\item Dans les treillis des \sgs, on a l'action de conjugaison de $G$ sur l'ensemble des \sgs par conjugaison.\\
Les orbites ponctuelles sont les \sgs distingués, cad les \sgs qui ne sont pas liés par une action de $\rho$ à d'autres \sgs.\\
Les orbites non ponctuelles sont les paquets de \sgs reliés par une action de $\rho$.\\
Le stabilisateur d'un \sg $H$ est l'ensemble des $g$ tq $gHg^{-1}=H$, donc c'est le normalisateur de $H$ dans $G$.\\
L'isomorphisme $G/Stab_G(x)\fong G\cdot x$ dit que l'indice du normalisateur $N_G(H)$ est égal au cardinal de l'orbite de $H$ sous l'action de conjugaison.

\item Un groupe $G$ est \textbf{simple} si $G\neq \{e\}$ et ses seuls \sgs distingués sont $\{e\}$ et $G$.\\
On verra plus tard que $\A_n$ est simple pour $n\geq 5$. Cela est lié au fait que l'équation du n-ième degré n'est pas résoluble par des radicaux pour $n\geq 5$ (voir $M1$).
\end{enumerate}
\end{remar}

\section{Groupes quotients}

Soient $G$ un groupe et $H \leq G$ un \sg. On note $\pi : G \rightarrow G/H,~g\mapsto gH$ la bijection canonique.

\begin{rappel}
\begin{itemize}
\item $\pi$ est surjective
\item $\forall g\in G$, on a $\pi(g)=gH$
\item Pour $g,g' \in G$, on a $\pi(g)=\pi(g')$ ssi $\exists h \in H$ tq $g'=gh$
\end{itemize}
\end{rappel}

\begin{thm}
\label{thm_comp_int}
On suppose que $H\vartriangleleft G$.\\
Il existe une unique loi de composition interne $*$ sur $G/H$ tq $(G/H,*)$ soit un groupe et 
$$\pi : G \rightarrow G/H$$
un morphisme de groupes.
\end{thm}

\begin{demo}
Soient $\alpha, \beta \in G/H$. Soient $x,y\in G$ tq $\pi (x)=\alpha$ et $\pi(y)=\beta$.\\
On définit 
$$ \alpha * \beta = \pi(x) * \pi(y)=\pi(xy)$$
et c'est la seule possibilité car $\pi$ doit être un morphisme de groupes.\\
Il faut vérifier que $\alpha * \beta$ est bien défini.\\
Soient $x',y'\in G$ tq $\pi(x')=\alpha$ et $\pi(y')=\beta$.\\
Il existe $h,k\in H$ tq $x'=xh$ et $y'=yk$. On a :
$$\pi(x'y')=\pi(xhyk)=\pi(xyy^{-1}hyk)=\pi(xy)$$
Ce qui montre que $\pi(xy)$ ne dépend que du choix des représentants $x$ et $y$ de $\alpha$ et $\beta$.\\
Montrons l'associativité : Soient $\pi (x),\pi(y),\pi(z)$ dans $G/H$. On a :
$$\begin{array}{lll}
(\pi(x)\pi(y))\pi(z)&=&\pi(xy)\pi(z)\\
&=&\pi(xyz)\\
&=&\pi(x)(\pi(y)\pi(z))
\end{array}$$
$\pi(e)$ est neutre car :
$$\pi(x)\pi(e)=\pi(xe)=\pi(x)=\pi(ex)=\pi(e)\pi(x)$$
et $\pi(x^{-1})$ est inverse de $\pi(x),~\forall x\in G$, car :
$$\pi(x)\pi(x^{-1})=\pi(xx^{-1})=\pi(e)=\pi(x^{-1}x)=\pi(x^{-1}x)=\pi(x^{-1})\pi(x)$$
\end{demo}

\begin{remar}
\begin{enumerate}
\item On suppose que $H\vartriangleleft G$. Alors la loi de composition sur $G/H$ vérifie :
\begin{itemize}
\item $H=\ker (\pi : G \rightarrow G/H)$
\item $eH=H$ est l'élément neutre
\item $\forall x, y\in G$ : $xH * yH = xyH$
\end{itemize}

\item Supposons que $H\leq G$ et $G/H$ a une structure de groupe telle que $\pi:G \rightarrow G/H$ est un morphisme. Alors $H=\ker (\pi)$ est forcément distingué dans $G$.
\end{enumerate}
\end{remar}


\begin{defi}
Supposons que $H\vartriangleleft G$. Le groupe $G/H$ du théorème \ref{thm_comp_int} est le \textbf{groupe quotient} de $G$ par $H$.
\end{defi}

\begin{cor}
Les \sgs distingués de $G$ sont exactement les noyaux des \mdgs $G\underset{\varphi}{\rightarrow} K$ de domaine $G$.
\end{cor}

\begin{exem}
\begin{enumerate}
\item Si $E$ est un \ev sur $\R$ et $F$ un \sev alors $(F,+)$ est un \sg distingué de $(E,+)$ (qui est commutatif).\\
Alors $E/F$ devient un \ev pour la multiplication par les scalaires définie par $\lambda.(v+F)=\lambda v+F$\\
Si $F'\subseteq E$ est un supplémentaire de $F$ ($E=F'\oplus F$), alors la composition :
$$F'~inj~E \rightarrow E/F$$
est un isomorphisme d'espaces vectoriels (exo !).\\
En particulier, si $\dim E < \infty$, alors :
$$\dim E/F = \dim E- \dim F$$

\item Soit $V=\{e,(1~2)(3~4),(1~3)(2~4), (1~4)(2~3)\}$. C'est un \sg de $\A_4$.\\
Alors $V$ est distingué dans $\A_4$ et le quotient $\A_4/V$ est d'ordre $12/4=3$.\\
Donc $\A_4/V$ est isomorphe à $\U_3$.\\
Il est instructif de voir la position de $V$ dans le treillis des \sgs de $\A_4$.\\

% ici graphe%%%%%%%%%%%%%%%%%%%%%%%%%%%%%%%%%%%%%%%%%%%%%%%%%%%%%%%%%%%%%%%%%%%%%%%%%%%%%%%%%%%%
%
\textcolor{red}{un beau et magnifique graphe fait par le très généreux Tristan arrivera prochainement sur vos ecrans/pdf (diagramme de Hasse de $\A_4$)}
%
% fin zone%%%%%%%%%%%%%%%%%%%%%%%%%%%%%%%%%%%%%%%%%%%%%%%%%%%%%%%%%%%%%%%%%%%%%%%%%%%%%%%%%%%%%%


On a :
$ N_{\A_4}(<(1~2)(3~4)>)=V$ est d'indice 3 dans $\A_4$\\
$ N_{\A_4}(<(1~2~3)>)=<(1~2~3)$ est d'indice 4 dans $\A_4$
\end{enumerate}
\end{exem}

\section{Passage au quotient des \mdgs}

\begin{thm}[Propriété universelle du groupe quotient]
\label{thm_prop_univ}
Soit $f:G\rightarrow K$ un \mdg. Soit $H\vartriangleleft G$. On suppose que $H\subseteq \ker (f)$.\\
Alors il existe un unique \mdg 
$$\bar{f} : G/H \rightarrow K$$
tq $f=\bar{f}\circ \pi$
\end{thm}

\begin{term}
On dit que $\bar{f}$ est obtenu à partir de $f$ par passage au quotient par $H$, ou que $\bar{f}$ est induit par $f$.
\end{term}

\begin{remar}
On a $Im(\bar{f})=Im(f)$ et $\ker(\bar{f})=\pi(\ker f)$
\end{remar}

\begin{demo}[Théorème \ref{thm_prop_univ}]
Comme $\pi$ est surjectif, $\bar{f}$ est unique. On définit pour $x\in G$, 
$$\bar{f}(\pi(x))=f(x)$$
On doit vérifier que si $\pi(x)=\pi(x')$, alors $f(x)=f(x')$. En effet, il existe $h\in H$ tq $x'=xh$ et on a donc :
$$f(x')=f(xh)=f(x)f(h)=f(x)$$
car $f(h)\in \ker f$.\\
Vérifions que $\bar{f}$ est bien un morphisme :\\
On a pour $x,y\in G$ :
$$\begin{array}{lll}
\bar{f}(\pi(x)\pi(y))&=&\bar{f}(\pi(xy))\\
&=& f(xy)\\
&=& f(x)f(y)\\
&=& \bar{f}(\pi(x))\bar{f}(\pi(y))
\end{array}$$
\end{demo}

\begin{thm}[Premier théorème d'isomorphisme]
Soit $f:G\rightarrow K$ un \mdg, alors $f$ induit un isomorphisme de groupes de $G/\ker(f)$ sur $Im(f)$.\\
En particulier, si $f$ est surjectif, alors $f$ induit un isomorphisme
$$G/\ker(f)\fong K$$
\end{thm}

\begin{demo}
$\bar{f}$ est bien définie par la propriété universelle du groupe quotient (théorème \ref{thm_prop_univ}). Clairement, $\bar{f}$ est surjectif.\\
Montrons que $\bar{f}$ est injectif : si on a $\bar{f}(\pi(x))=\bar{f}(\pi(x'))$, alors $f(x)=f(x')$, donc $x^{-1}x'\in \ker(f)$ et donc $\pi(x)=\pi(x')$ dans $G/\ker(f)$.
\end{demo}

\begin{remar}
Ce théorème est important car il relie le groupe quotient, qui a priori est difficile à comprendre, avec un \sg, qui est plus facile à comprendre.
\end{remar}

\begin{exem}
\begin{enumerate}
\item Soit $f:\Z \rightarrow \U_n,~k\mapsto e^{\frac{2\pi ik}{n}}$, alors $f$ est surjectif de noyau $n\Z \subseteq \Z$.\\
Donc $f$ induit un isomorphisme $\Z/n\Z \fong \U_n$.
\item Soit $\epsilon : \sigma_n \rightarrow \{\pm 1\}$ la signature, alors $\epsilon$ est surjectif de noyau $\A_n$.\\
Donc $\epsilon$ induit un \isom 
$$\sigma_n/\A_n \fong \{\pm 1\}$$
En particulier, on a $|\A_n| = \fracun{2}(n!)$.
\end{enumerate}
\end{exem}

\begin{lemme}
Soient $G$ un groupe fini et $p$ le plus petit diviseur premier de $|G|$.\\
Soit $H\subseteq G$ un \sg d'indice $p$. Alors $H$ est distingué.
\end{lemme}

\begin{remar}
Pour $p=2$, on retrouve le fait qu'un \sg d'indice 2 est toujours distingué.
\end{remar}

\begin{demo}
Faisons agir $G$ sur $G/H$ par translation à gauche :
$$g.xH := gxH,~\forall g\in G,\forall x\in G$$
Comme $|G/H|=p$, cela définit un \mdg $f:G\rightarrow \sigma_p$. Ce morphisme est non trivial car l'action est transitive.\\
Notons $K$ son image. Alors $|K|$ divise à la fois $|G|$ et $|\sigma_p|=p!$. Donc il divise $pgcd(|G|,p!)=p$. Donc $|K|=p$.\\
Comme $G/\ker(f)\fong K$, l'indice de $\ker (f)$ dans $G$ est $p$. Or les éléments de $\ker(f)$ laissent fixe $eH$. Donc $\ker (f)\subseteq H$. Comme les deux sont d'indice $p$ dans $G$, ils sont égaux et $H=\ker (f)$ est distingué.
\end{demo}

\begin{thm}[Deuxième théorème d'\isom]
Soient $G$ un groupe, $H$ un \sg distingué et $K\subseteq G$ un \sg. Alors $HK=\{hk| j\in H, k\in K\}$ est un \sg de $G$ et on a un \isom de groupes 
$$K/H\cap K \fong HK/H$$
\end{thm}

\begin{exem}
Si $F,G$ sont des \sevs d'un \ev $E$, on a 
$$F/F\cap G \fong F+G/F$$
\end{exem}

\begin{demo}
Montrons que $HK$ est un \sg :\\
$e=e.e \in HK$. Si $x,x'\in H$ et $y,y' \in K$, on a :
$$(xy)^{-1}=y^{-1}x^{-1}=y^{-1}x^{-1}yy^{-1}\in HK$$
et 
$$xyx'y'=xyx'y^{-1}yy'\in HK$$
La composée 
$$K~inj~HK~surj~HK/H$$
est un \mdg surjectif dont le noyau est $K\cap H$. Par le premier théorème d'\isom, on obtient l'\isom $K/H\cap K \fong HK/H$.
\end{demo}

\begin{thm}[Troisième théorème d'\isom]
Soient $H\subseteq K$ deux \sgs distingués d'un groupe $G$. Alors, on a 
$$G/K \fong (G/H) / (G/K)$$
\end{thm}

\begin{demo}
La composée des surjections canoniques 
$$G\rightarrow G/H \rightarrow (G/H)/(G/K)$$
est un \mdg surjectif de noyau $K$. Le premier théorème d'\isom nous donne l'\isom 
$$G/K \rightarrow (G/H)/(K/H)$$
\end{demo}

















\end{document}